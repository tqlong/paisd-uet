\documentclass{beamer}

% Vietnamese support
\usepackage{fontenc}

\newcommand{\ul}[1]{\emph{#1}}

% Beamer theme
\usetheme{Berlin}

\title{Thực hành phát triển hệ thống Trí tuệ nhân tạo\\
Phân tích thiết kế hướng đối tượng (OOAD)
}
\author{Trần Quốc Long}
\institute{Trường ĐH Công nghệ, ĐHQG Hà Nội}
\date{\today}

\begin{document}

\begin{frame}
    \titlepage
\end{frame}

\begin{frame}{Mục lục}
    \tableofcontents
\end{frame}

\section{Lập trình hướng đối tượng}

\begin{frame}{Các nguyên tắc trong lập trình hướng đối tượng}
    \begin{columns}
    \column{0.5\textwidth}
    \begin{itemize}
        \item \ul{Trừu tượng} (Abstraction)
        \item \ul{Đóng gói} (Encapsulation)
        \item \ul{Mô-đun} (Modularity)
        \item \ul{Phân cấp} (Hierarchy)
    \end{itemize}
    \column{0.5\textwidth}
    \begin{itemize}
        \item \ul{Đa hình} (Polymorphism)
        \item \ul{Thừa kế} (Inheritance)
        \item \ul{Hợp thành} (Composition)
        \item \ul{Giao tiếp} (Delegation)
        \item \ul{Phân loại} (Subtyping)
        \item \ul{Ẩn dữ liệu} (Data Hiding)
        \item \ul{Tách biệt quan tâm} (Separation of Concerns)
    \end{itemize}
    \end{columns}
\end{frame}

\begin{frame}{Trừu tượng hoá}
    \begin{figure}
        \centering
        \includegraphics[width=\textwidth]{oop-vs-pp}
    \end{figure}
\end{frame}

\begin{frame}{Trừu tượng và đa hình}
    \begin{figure}
        \centering
        \includegraphics[width=0.6\textwidth]{java-abstraction}
    \end{figure}
\end{frame}

\begin{frame}{Trừu tượng và đa hình}
    \begin{figure}
        \centering
        \includegraphics[width=0.4\textwidth]{java-abstraction2}
    \end{figure}
\end{frame}

\begin{frame}{Đóng gói}
    \begin{figure}
        \centering
        \includegraphics[width=0.9\textwidth]{encapsulation}
    \end{figure}
\end{frame}

\begin{frame}{Đóng gói và Ẩn dữ liệu}
    \begin{figure}
        \centering
        \includegraphics[width=0.9\textwidth]{encapsulation2}
    \end{figure}
\end{frame}

\begin{frame}{Đóng gói và Ẩn dữ liệu}
    \begin{figure}
        \centering
        \includegraphics[width=0.6\textwidth]{naked1}
    \end{figure}
    \begin{figure}
        \centering
        \includegraphics[width=0.6\textwidth]{naked2}
    \end{figure}
\end{frame}

\begin{frame}{Mô-đun}
    \begin{figure}
        \centering
        \includegraphics[width=\textwidth]{modularity}
    \end{figure}
\end{frame}

\begin{frame}{Phân cấp}
    \begin{figure}
        \centering
        \includegraphics[width=0.9\textwidth]{hierarchy}
    \end{figure}
\end{frame}

\begin{frame}{Thừa kế và Hợp thành}
    \begin{figure}
        \centering
        \includegraphics[width=0.9\textwidth]{inheritance-versus-composition}
    \end{figure}
\end{frame}

\begin{frame}{Thừa kế và Hợp thành}
%% create table comparion of inheritance and composition
%% when, why, how
\begin{table}
    \centering
    \begin{tabular}{|p{2cm}|p{3.5cm}|p{3.5cm}|}
        \hline
        & \textbf{Thừa kế} & \textbf{Hợp thành} \\
        \hline
        \textbf{Khi nào} & Quan hệ ``is-a'' \newline ràng buộc chặt 
        & Quan hệ ``has-a'' \newline ràng buộc lỏng\\
        \hline
        \textbf{Khi nào không} &
            Chỉ dùng một phần chức năng của lớp cha\newline
            Lớp cha thay đổi thường xuyên\newline
            Cần thừa kế từ nhiều lớp cha
        &  Thừa kế giảm nhiều trùng lặp hơn \newline
        Chức năng cần được thừa kế và nạp chồng \\
        \hline
        \textbf{Tại sao} & Tái sử dụng mã nguồn & Tái sử dụng chức năng \\
        \hline
    \end{tabular}
\end{table}

\end{frame}

\section{Nguyên tắc SOLID}

\begin{frame}{Các nguyên tắc SOLID}
    \begin{columns}
        \column{0.3\textwidth}
        \begin{figure}
            \centering
            \includegraphics[width=0.8\textwidth]{../images/books/agile-martin}
        \end{figure}
        \column{0.7\textwidth}
            \only<1>{
                \begin{block}{}
                    Trong Agile, thái độ của nhà phát triển đối với
                    thiết kế phần mềm giống như thái độ của các bác
                    sĩ phẫu thuật đối với quy trình vô trùng.
                    Quy trình vô trùng điều kiện tiên quyết của phẫu
                    thuật. Nếu không có nó, nguy cơ nhiễm trùng sẽ
                    quá cao để chịu đựng. Các nhà phát triển Agile
                    cũng cảm thấy tương tự đối với thiết kế của họ.\\
                    \textbf{Robert Martin}, \emph{Agile Software Development. Principles, Patterns, and Practices}
                \end{block}
            }
    \end{columns}
\end{frame}

\begin{frame}{Các nguyên tắc SOLID}
    \begin{itemize}
        \item \textbf{\color{red}S}ingle Responsibility Principle - đơn trách nhiệm
        \item \textbf{\color{red}O}pen Close Principle - mở đóng
        \item \textbf{\color{red}L}iskov Substitution Principle - thay thế Liskov
        \item \textbf{\color{red}I}nterface Segregation Principle - phân chia giao diện
        \item \textbf{\color{red}D}ependency Inversion Principle - đảo ngược phụ thuộc
    \end{itemize}
\end{frame}

\begin{frame}{Nguyên tắc đơn trách nhiệm}
    \begin{itemize}
        \item Một lớp/mô-đun chỉ nên có một lý do để thay đổi
        \item Mỗi lớp/mô-đun chỉ nên thực hiện một chức năng cụ thể
        và phải đóng gói chức năng đó
        \item Mỗi lớp/mô-đun chỉ nên thực hiện một vai trò
    \end{itemize}

    \begin{table}
        \centering
        \begin{tabular}{|p{4.5cm}|p{4.5cm}|}
            \hline\vspace{-0.5cm}
            \begin{center}\textbf{Nên dùng}\end{center} 
            & \vspace{-0.5cm}\begin{center}\textbf{Không nên dùng}\end{center} \\
            \hline
            \vspace{-0.3cm}
            \begin{itemize}
                \item Lớp có nhiều chức năng không liên quan
                \item Lớp quá lớn
                \item Lớp có bộ phận hay thay đổi
            \end{itemize}
            & \vspace{-0.3cm}\begin{itemize}
                \item Phân tách lớp khiến code phức tạp hơn
                \item Phạm vi của lớp nhỏ, các thành phần liên quan mạnh
                \item Hiệu suất giảm quá nhiều
            \end{itemize} \\
            \hline
        \end{tabular}
    \end{table}
\end{frame}

\begin{frame}{Nguyên tắc đơn trách nhiệm}
    \begin{figure}
        \centering
        \includegraphics[width=\textwidth]{srp}
    \end{figure}
\end{frame}

\begin{frame}{Nguyên tắc mở đóng}
    \Large Một lớp phải \textbf{\color{red}mở} để mở rộng
    nhưng \textbf{\color{red}đóng} đối với sửa đổi
    \vspace{-0.5cm}
    \begin{figure}
        \centering
        \includegraphics[width=0.7\textwidth]{ocp}
    \end{figure}
\end{frame}

\begin{frame}{Nguyên tắc thay thế Liskov}
    Nếu với mỗi đối tượng $o_1$ thuộc lớp $S$ 
    có đối tượng $o_2$ thuộc lớp $T$ sao cho mọi chương trình
    $P$ viết dựa trên $T$ mà hành vi của $P$ không thay đổi
    khi $o_1$ thay thế $o_2$ thì $S$ là lớp con của $T$
    -- \textbf{Barbara Liskov}
    % \vspace{-0.3cm}
    \begin{figure}
        \centering
        \includegraphics[width=0.7\textwidth]{lsp}
    \end{figure}
\end{frame}

\begin{frame}{Duck Typing}
    \Large Nếu nó trông như vịt, bơi như vịt và 
    kêu như vịt, thì chắc nó là vịt
    \begin{figure}
        \centering
        \includegraphics[width=0.7\textwidth]{duck}
    \end{figure}
\end{frame}

\begin{frame}{Nguyên tắc phân chia giao diện}
    \Large Đối tượng khách (client) không nên bị ép buộc
    phụ thuộc vào các phương thức mà nó không sử dụng
    -- \textbf{Robert Martin}
    \begin{figure}
        \centering
        \includegraphics[width=0.5\textwidth]{isp}
    \end{figure}
\end{frame}

\begin{frame}{Nguyên tắc đảo ngược phụ thuộc}
    (a) Các module cấp cao không nên phụ thuộc vào
    các module cấp thấp. Cả hai nên phụ thuộc vào các giao diện
    (b) Các giao diện không nên phụ thuộc vào chi tiết.
    Chi tiết nên phụ thuộc vào giao diện
    \begin{figure}
        \centering
        \includegraphics[width=\textwidth]{dip}
    \end{figure}
\end{frame}

% \begin{frame}{Lập trình cực đoan (XP)}
%     \begin{figure}
%         \centering
%         \includegraphics[width=0.65\textwidth]{xp}
%     \end{figure}
% \end{frame}

% \begin{frame}{Chi phí của bug}
%     \begin{figure}
%         \centering
%         \includegraphics[width=0.9\textwidth]{cost-of-bug}
%     \end{figure}
% \end{frame}

% \section{Các yếu tố chất lượng của thiết kế}

% \begin{frame}{Tính đơn giản}
%     \begin{itemize}
%         \item \ul{\color{red}Đầu tư} nhỏ nhất có thể vào thiết kế trước khi
%         nhận được lợi nhuận từ nó. -- \textbf{Kent Beck}
%         \item Thiết kế chứa \ul{\color{red}sự phức tạp không cần thiết} khi
%         có các yếu tố không hữu ích. Điều này thường xảy ra khi
%         ta dự đoán các thay đổi có thể có đối với yêu cầu và tìm cách
%         xử lý những thay đổi tiềm năng đó. -- \textbf{Robert Martin}        
%         \item Không làm chủ sự phức tạp của phần mềm 
%         có hậu quả là các dự án bị trễ, vượt ngân sách và thiếu sót
%         so với yêu cầu. Tình trạng này gọi là
%         \ul{\color{red}khủng hoảng phần mềm}, nhưng thẳng thắn mà nói,
%         một căn bệnh kéo dài như vậy phải được gọi là
%         trạng thái bình thường. -- \textbf{Grady Booch}
%     \end{itemize}
% \end{frame}

% \begin{frame}{Tính đơn giản}
%     \begin{figure}
%         \centering
%         \includegraphics[width=0.9\textwidth]{complexity1}
%     \end{figure}
% \end{frame}

% \begin{frame}{Tính đơn giản}
%     \begin{figure}
%         \centering
%         \includegraphics[width=0.8\textwidth]{complexity2}
%     \end{figure}
% \end{frame}

% \begin{frame}{Tính đơn giản}
%     \begin{block}{}
%         Đức tính chính của một kiến trúc sư là khả năng
%         \ul{giảm bớt sự phức tạp}. Một kiến trúc sư giỏi
%         không bao giờ tự hào về một sơ đồ phức tạp. Thay vào
%         đó, họ sẽ tự hào về một bản vẽ đơn giản và dễ hiểu
%         với vài hình chữ nhật mà giải thích hoàn hảo một
%         ứng dụng đa tầng. Việc này thực sự khó. Đó
%         là nơi mà tư duy kiến trúc thực sự tỏa sáng
%         -- \textbf{Yegor Bugayenko}
%     \end{block}
% \end{frame}

% \begin{frame}{Tính không trùng lặp}
%     \begin{itemize}
%         \item Khi có \ul{\color{red}mã dư thừa} trong hệ thống, việc thay
%         đổi hệ thống trở nên khó khăn. Các lỗi được tìm thấy
%         phải được sửa trong mọi lần lặp lại. Tuy nhiên,
%         vì mỗi lần lặp lại hơi khác nhau, nên việc sửa
%         lỗi không phải lúc nào cũng giống nhau -- \textbf{Robert Martin}
%     \end{itemize}
% \end{frame}

% \begin{frame}{Tính mô-đun hóa}
%     \begin{itemize}
%         \item Một thiết kế là \ul{\color{red}bất động} nếu nó
%         chứa các thành phần
%         được dùng trong các hệ thống khác, nhưng nỗ lực và
%         rủi ro để tách các phần đó ra khỏi hệ thống ban đầu
%         là quá lớn -- \textbf{Robert Martin}
%         \item Một thiết kế là \ul{\color{red}cứng} nếu một
%         thay đổi đơn lẻ gây ra một loạt các thay đổi tiếp
%         theo trong các module phụ thuộc.
%         Càng nhiều module phải thay đổi,
%         thiết kế càng \ul{cứng} -- \textbf{Robert Martin}
%     \end{itemize}
% \end{frame}

% \begin{frame}{Tài liệu tham khảo}
%     \begin{columns}
%         \column{0.5\textwidth}
%         \begin{figure}
%             \centering
%             \includegraphics[width=0.4\textwidth]{../images/books/rup}
%             \caption{Per Kroll và cộng sự, \emph{A Practitioner's Guide to the RUP}}
%         \end{figure}
%         \column{0.5\textwidth}
%         \begin{figure}
%             \centering
%             \includegraphics[width=0.4\textwidth]{../images/books/agile-martin}
%             \caption{Robert Martin,
%             \emph{Agile Software Development. Principles, Patterns, and Practices}}
%         \end{figure}
%     \end{columns}
% \end{frame}

% \begin{frame}{Tài liệu tham khảo}
%     \begin{columns}
%         \column{0.5\textwidth}
%         \begin{figure}
%             \centering
%             \includegraphics[width=0.4\textwidth]{../images/books/clean-architecture}
%             \caption{Robert Martin, \emph{Clean Architecture: A Craftsman's Guide to Software Structure and Design}}
%         \end{figure}
%         \column{0.5\textwidth}
%         \begin{figure}
%             \centering
%             \includegraphics[width=0.4\textwidth]{../images/books/code-complete}
%             \caption{Steve McConnell, \emph{Code Complete}}
%         \end{figure}
%     \end{columns}
% \end{frame}

% \begin{frame}{Tài liệu tham khảo}
%     \begin{columns}
%         \column{0.5\textwidth}
%         \begin{figure}
%             \centering
%             \includegraphics[width=0.4\textwidth]{../images/books/brooks}
%             \caption{Frederick Brooks Jr., \emph{Mythical Man-Month, The: Essays on Software Engineering}}
%         \end{figure}
%         \column{0.5\textwidth}
%         \begin{figure}
%             \centering
%             \includegraphics[width=0.4\textwidth]{../images/books/pragmatic}
%             \caption{David Thomas và cộng sự, \emph{The Pragmatic Programmer: Your Journey To Mastery}}
%         \end{figure}
%     \end{columns}
% \end{frame}

\end{document}
