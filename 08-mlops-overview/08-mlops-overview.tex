\documentclass{beamer}

% Vietnamese support
\usepackage{fontenc}
\usepackage{qrcode}

\newcommand{\ul}[1]{\emph{#1}}

% Beamer theme
\usetheme{Berlin}

\title{Thực hành phát triển hệ thống Trí tuệ nhân tạo\\
Tổng quan về MLOps
}
\author{Trần Quốc Long}
\institute{Trường ĐH Công nghệ, ĐHQG Hà Nội}
\date{\today}

\begin{document}
\begin{frame}
    \titlepage
\end{frame}

\begin{frame}{Mục lục}
    \tableofcontents
\end{frame}

\section{Quy trình phát triển hệ thống TTNT}

\begin{frame}{Quy trình phát triển hệ thống TTNT}
    \begin{columns}
        \column{0.1\textwidth}
        \begin{figure}
            \centering
            % \includegraphics[width=0.8\textwidth]{../images/ai-development-process}
        \end{figure}
        \column{0.9\textwidth}
        \begin{enumerate}
            \item Thu thập và xử lý dữ liệu
            \item Lựa chọn thuật toán và xây dựng mô hình
            \item Huấn luyện và đánh giá mô hình
            \item Triển khai và vận hành mô hình
            \item Theo dõi, cải thiện và bảo trì hệ thống
        \end{enumerate}
    \end{columns}
\end{frame}

\begin{frame}{Bước 1: Thu thập và xử lý dữ liệu}
    \begin{columns}
        \column{0.1\textwidth}
        \begin{figure}
            \centering
            % \includegraphics[width=0.8\textwidth]{../images/faces/alan-kay}
        \end{figure}
        \column{0.9\textwidth}
        \begin{enumerate}
            \item Tìm kiếm và thu thập dữ liệu
            \item Tiền xử lý dữ liệu (làm sạch, chuyển đổi, chuẩn hóa)
            \item Khám phá và phân tích dữ liệu
            \item Chọn lựa đặc trưng
            \item Lưu trữ dữ liệu
            \item Tạo tập dữ liệu huấn luyện, kiểm thử và đánh giá
        \end{enumerate}
    \end{columns}
\end{frame}

\begin{frame}{Bước 2: Lựa chọn thuật toán và xây dựng mô hình}
    \begin{columns}
        \column{0.1\textwidth}
        \begin{figure}
            \centering
            % \includegraphics[width=0.8\textwidth]{../images/faces/paul-graham}
        \end{figure}
        \column{0.9\textwidth}
        \begin{enumerate}
            \item Lựa chọn thuật toán học máy: Dựa trên đặc điểm dữ liệu và bài toán
            \begin{itemize}
                \item Học có giám sát: Phân loại, hồi quy, \dots
                \item Học không giám sát: Phân cụm, giảm chiều, \dots
                \item Học tăng cường: Q-learning, Deep Q-learning, \dots
            \end{itemize}
            \item Xây dựng mô hình: Thiết kế kiến trúc mô hình
            \begin{itemize}
                \item Mạng nơ-ron
                \item Rừng ngẫu nhiên
                \item Hồi quy logistic
                \item \dots
            \end{itemize}
        \end{enumerate}
    \end{columns}
\end{frame}

\begin{frame}{Bước 3: Huấn luyện và đánh giá mô hình}
    \begin{columns}
        \column{0.1\textwidth}
        \begin{figure}
            \centering
            % \includegraphics[width=0.8\textwidth]{../images/faces/linus-torvalds}
        \end{figure}
        \column{0.9\textwidth}
        \begin{itemize}
            \item Huấn luyện mô hình: Sử dụng tập dữ liệu huấn luyện để tối ưu hóa tham số mô hình
            \begin{itemize}
                \item Chọn hàm mất mát
                \item Chọn thuật toán tối ưu hóa
                \item Điều chỉnh siêu tham số siêu (hyperparameter)
                \item Sử dụng kỹ thuật tăng cường (data augmentation)
            \end{itemize}
        \end{itemize}
    \end{columns}
\end{frame}

\begin{frame}{Bước 3: Huấn luyện và đánh giá mô hình}
    \begin{columns}
        \column{0.1\textwidth}
        \begin{figure}
            \centering
            % \includegraphics[width=0.8\textwidth]{../images/faces/linus-torvalds}
        \end{figure}
        \column{0.9\textwidth}
        \begin{itemize}
            \item Đánh giá mô hình: Sử dụng tập dữ liệu kiểm thử để đánh giá hiệu suất mô hình
            \begin{itemize}
                \item Tỉ lệ chính xác, độ chính xác, độ hồi tưởng, F1-score, độ nhạy, độ đặc hiệu, ma trận nhầm lẫn
                \item Đường đặc trưng nhận diện (ROC), diện tích dưới đường đặc trưng nhận diện (AUC)
                \item Kiểm tra chéo (cross-validation)
                \item Sử dụng kỹ thuật kiểm tra A/B (A/B testing)
            \end{itemize}
        \end{itemize}
    \end{columns}
\end{frame}

\begin{frame}{Công cụ hỗ trợ huấn luyện và đánh giá}
    \begin{columns}
        \column{0.1\textwidth}
        \begin{figure}
            \centering
            % \includegraphics[width=0.8\textwidth]{../images/faces/linus-torvalds}
        \end{figure}
        \column{0.9\textwidth}
        \begin{itemize}
            \item Quản lý cấu hình: Hydra (https://hydra.cc/)
            \begin{itemize}
                \item Quản lý tham số huấn luyện
                \item Quản lý tham số thí nghiệm
                \item Kết nối với các công cụ khác như Weights \& Biases, TensorBoard
            \end{itemize}
            \item Quản lý thí nghiệm: Weights \& Biases (https://wandb.ai/), MLFlow (https://mlflow.org/)
            \begin{itemize}
                \item Theo dõi quá trình huấn luyện
                \item Lưu trữ mô hình
                \item So sánh các thí nghiệm
                \item Tạo báo cáo
                \item Chia sẻ kết quả
            \end{itemize}
        \end{itemize}
    \end{columns}
\end{frame}

\begin{frame}{Bước 4: Triển khai và vận hành mô hình}
    \begin{columns}
        \column{0.1\textwidth}
        \begin{figure}
            \centering
            % \includegraphics[width=0.8\textwidth]{../images/faces/linus-torvalds}
        \end{figure}
        \column{0.9\textwidth}
        \begin{itemize}
            \item Triển khai mô hình: Đưa mô hình vào môi trường sản xuất
            \begin{itemize}
                \item Web, REST API, gRPC
                \item Dịch vụ đám mây (AWS, Google Cloud, Azure)
                \item Mobile
            \end{itemize}
            \item Vận hành mô hình: Theo dõi và bảo trì mô hình
            \begin{itemize}
                \item Đảm bảo ổn định (Docker, Kubernetes, Triton)
                \item Hiệu suất (DeepStream, TensorRT, ONNX)
            \end{itemize}
        \end{itemize}
    \end{columns}
\end{frame}

\begin{frame}{Bước 5: Theo dõi, cải thiện và bảo trì hệ thống}
    \begin{columns}
        \column{0.1\textwidth}
        \begin{figure}
            \centering
            % \includegraphics[width=0.8\textwidth]{../images/faces/linus-torvalds}
        \end{figure}
        \column{0.9\textwidth}
        \begin{itemize}
            \item Theo dõi mô hình: Theo dõi hiệu suất mô hình trong môi trường sản xuất
            \begin{itemize}
                \item Đánh giá độ chính xác
                \item Theo dõi độ trễ và băng thông
                \item Theo dõi tài nguyên hệ thống (CPU, GPU, RAM)
                \item Phát hiện "data drift" và "concept drift"
            \end{itemize}
            \item Cải thiện mô hình: Cập nhật mô hình dựa trên dữ liệu mới
            \begin{itemize}
                \item Tinh chỉnh tham số mô hình
                \item A/B testing so sánh mô hình mới và cũ
                \item Quản lý phiên bản mô hình
                \item Tự động hóa quy trình phát triển mô hình
            \end{itemize}
        \end{itemize}
    \end{columns}
\end{frame}

\section{MLOps}

\begin{frame}{Vai trò của MLOps}
    \begin{columns}
        \column{0.2\textwidth}
        \begin{figure}
            \centering
            % \includegraphics[width=0.8\textwidth]{../images/faces/linus-torvalds}
        \end{figure}
        \column{0.8\textwidth}
        \begin{itemize}
            \item Tự động hóa
            \begin{itemize}
                \item Tự động hóa quy trình phát triển mô hình
                \item Giảm thiểu công việc thủ công, lỗi do con người
            \end{itemize}
            \item Khả năng lặp lại
            \begin{itemize}
                \item Đảm bảo kết quả nhất quán
                \item Dễ dàng so sánh các mô hình, theo dõi các thay đổi
            \end{itemize}
        \end{itemize}
    \end{columns}
\end{frame}

\begin{frame}{Lợi ích của MLOps}
    \begin{columns}
        \column{0.2\textwidth}
        \begin{figure}
            \centering
            % \includegraphics[width=0.8\textwidth]{../images/faces/linus-torvalds}
        \end{figure}
        \column{0.8\textwidth}
        \begin{itemize}
            \item Dễ mở rộng
            \begin{itemize}
                \item Hỗ trợ quản lý nhiều mô hình và phiên bản
                \item Hỗ trợ phát triển mô hình trên nhiều môi trường khác nhau
            \end{itemize}
            \item Hợp tác hiệu quả
            \begin{itemize}
                \item Hỗ trợ làm việc nhóm: Data scientists, engineers, DevOps
            \end{itemize}
        \end{itemize}
    \end{columns}
\end{frame}

\begin{frame}{Công cụ MLOps}

        \begin{itemize}
            \item Thu thập dữ liệu: Apache Kafka, Apache NiFi
            \item Xây dựng mô hình: TensorFlow, PyTorch, Scikit-learn
            \item Triển khai mô hình: TensorFlow Serving, Triton Inference Server
            \item Theo dõi mô hình: Prometheus, Grafana
            \item Quản lý mô hình: MLflow, DVC
            \item Tự động hóa quy trình: Kubeflow, Airflow
            \item Quản lý tài nguyên: Kubernetes, Docker
            \item Quản lý phiên bản: Git, DVC
            \item Quản lý cấu hình: Hydra
            \item Quản lý thí nghiệm: Weights \& Biases, Comet.ml
        \end{itemize}

\end{frame}

\end{document}
