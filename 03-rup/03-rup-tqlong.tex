\documentclass{beamer}

% Vietnamese support
\usepackage{fontenc}

\newcommand{\ul}[1]{\emph{#1}}

% Beamer theme
\usetheme{Berlin}

\title{Thực hành phát triển hệ thống Trí tuệ nhân tạo\\
RUP, Agile/XP
}
\author{Trần Quốc Long}
\institute{Trường ĐH Công nghệ, ĐHQG Hà Nội}
\date{\today}

\begin{document}

\begin{frame}
    \titlepage
\end{frame}

\begin{frame}{Mục lục}
    \tableofcontents
\end{frame}

\section{Các phương pháp phát triển phần mềm}

\begin{frame}{Các phương pháp phát triển}
    \begin{itemize}
        \item Quy trình thác nước (Waterfall/SDLC)
        \item Quy trình phát triển thống nhất (RUP)
        \item Khung phần mềm của Microsoft (MSF)
        \item Thư viện cơ sở hạ tầng CNTT (ITIL)
        \item Phong cách Agile và các phương pháp khác
    \end{itemize}
\end{frame}

\begin{frame}{Quy trình thác nước}
    \begin{figure}
        \centering
        \includegraphics[width=0.8\textwidth]{waterfall}
    \end{figure}
\end{frame}

\begin{frame}{Vòng đời phần mềm (SDLC)}
    \begin{figure}
        \centering
        \includegraphics[width=0.8\textwidth]{sdlc}
    \end{figure}
\end{frame}

\section{RUP}

\begin{frame}{Quy trình phát triển thống nhất (RUP)}
    Lặp và tăng trưởng
    \begin{figure}
        \centering
        \includegraphics[width=0.8\textwidth]{rup}
    \end{figure}
\end{frame}

\begin{frame}{Lặp (Iterative)}
    \begin{figure}
        \centering
        \includegraphics[width=0.8\textwidth]{iterative}
    \end{figure}
\end{frame}

\begin{frame}{Tăng trưởng (Incremental)}
    \begin{figure}
        \centering
        \includegraphics[width=0.8\textwidth]{i-and-i}
    \end{figure}
\end{frame}

\begin{frame}{Vai trò trong RUP}
    \begin{figure}
        \centering
        \includegraphics[width=0.7\textwidth]{rup-architect}
    \end{figure}
    \begin{figure}
        \centering
        \includegraphics[width=0.7\textwidth]{rup-designer}
    \end{figure}
\end{frame}

%% Microsoft
\begin{frame}{Khung phần mềm của Microsoft (MSF)}
    \begin{figure}
        \centering
        \includegraphics[width=0.9\textwidth]{msf}
    \end{figure}
\end{frame}

\begin{frame}{Thư viện cơ sở hạ tầng CNTT (ITIL)}
    \begin{figure}
        \centering
        \includegraphics[width=0.5\textwidth]{itil}
    \end{figure}
\end{frame}

\begin{frame}{Các phương pháp khác}
    \begin{columns}
        \column{0.5\textwidth}
        \begin{figure}
            \centering
            \includegraphics[width=0.7\textwidth]{spiral}
            \caption{Spiral}
        \end{figure}
        \column{0.5\textwidth}
        \begin{figure}
            \centering
            \includegraphics[width=0.5\textwidth]{vmodel}
            \caption{V-Model}
        \end{figure}
        \begin{figure}
            \centering
            \includegraphics[width=0.5\textwidth]{rad}
            \caption{Rapid Application Development}
        \end{figure}
    \end{columns}
\end{frame}

\section{Agile}

\begin{frame}{Agile}
    \begin{columns}
        \column{0.3\textwidth}
        \begin{figure}
            \centering
            \includegraphics[width=0.8\textwidth]{../images/books/agile-martin}
        \end{figure}
        \column{0.7\textwidth}
            \only<1>{
                \begin{block}{}
                    Trong Agile, bức tranh tổng thể phát triển
                    cùng với phần mềm. Với mỗi lần lặp,
                    thiết kế của hệ thống được cải thiện 
                    tốt nhất có thể tại thời điểm đó.
                    Không cần dành nhiều thời gian để nhìn trước
                    các yêu cầu và nhu cầu trong tương lai.\\
                    \textbf{Robert Martin}, \emph{Agile Software Development. Principles, Patterns, and Practices}
                \end{block}
            }
    \end{columns}
\end{frame}

\begin{frame}{Quy trình Agile}
    \begin{figure}
        \centering
        \includegraphics[width=0.7\textwidth]{agile}
    \end{figure}
\end{frame}

\begin{frame}{Công cụ Kanban}
    \begin{figure}
        \centering
        \includegraphics[width=0.9\textwidth]{kanban}
    \end{figure}
\end{frame}

\begin{frame}{Khung Scrum}
    \begin{figure}
        \centering
        \includegraphics[width=\textwidth]{scrum}
    \end{figure}
\end{frame}

\begin{frame}{Lập trình cực đoan (XP)}
    \begin{columns}
        \column{0.3\textwidth}
        \begin{figure}
            \centering
            \includegraphics[width=0.8\textwidth]{../images/books/xp}
        \end{figure}
        \column{0.7\textwidth}
            \only<1>{
                \begin{block}{}
                    Liên tục tinh chỉnh thiết kế
                    của hệ thống, bắt đầu từ một khởi đầu rất đơn giản.
                    Loại bỏ bất kỳ sự linh hoạt nào không hiệu quả.\\
                    \textbf{Kent Beck}, \emph{Extreme Programming Explained: Embrace Change}
                \end{block}
            }
    \end{columns}
\end{frame}

\begin{frame}{Lập trình cực đoan (XP)}
    \begin{figure}
        \centering
        \includegraphics[width=0.65\textwidth]{xp}
    \end{figure}
\end{frame}

\begin{frame}{Chi phí của bug}
    \begin{figure}
        \centering
        \includegraphics[width=0.9\textwidth]{cost-of-bug}
    \end{figure}
\end{frame}

\section{Các yếu tố chất lượng của thiết kế}

\begin{frame}{Tính đơn giản}
    \begin{itemize}
        \item \ul{\color{red}Đầu tư} nhỏ nhất có thể vào thiết kế trước khi
        nhận được lợi nhuận từ nó. -- \textbf{Kent Beck}
        \item Thiết kế chứa \ul{\color{red}sự phức tạp không cần thiết} khi
        có các yếu tố không hữu ích. Điều này thường xảy ra khi
        ta dự đoán các thay đổi có thể có đối với yêu cầu và tìm cách
        xử lý những thay đổi tiềm năng đó. -- \textbf{Robert Martin}        
        \item Không làm chủ sự phức tạp của phần mềm 
        có hậu quả là các dự án bị trễ, vượt ngân sách và thiếu sót
        so với yêu cầu. Tình trạng này gọi là
        \ul{\color{red}khủng hoảng phần mềm}, nhưng thẳng thắn mà nói,
        một căn bệnh kéo dài như vậy phải được gọi là
        trạng thái bình thường. -- \textbf{Grady Booch}
    \end{itemize}
\end{frame}

\begin{frame}{Tính đơn giản}
    \begin{figure}
        \centering
        \includegraphics[width=0.9\textwidth]{complexity1}
    \end{figure}
\end{frame}

\begin{frame}{Tính đơn giản}
    \begin{figure}
        \centering
        \includegraphics[width=0.8\textwidth]{complexity2}
    \end{figure}
\end{frame}

\begin{frame}{Tính đơn giản}
    \begin{block}{}
        Đức tính chính của một kiến trúc sư là khả năng
        \ul{giảm bớt sự phức tạp}. Một kiến trúc sư giỏi
        không bao giờ tự hào về một sơ đồ phức tạp. Thay vào
        đó, họ sẽ tự hào về một bản vẽ đơn giản và dễ hiểu
        với vài hình chữ nhật mà giải thích hoàn hảo một
        ứng dụng đa tầng. Việc này thực sự khó. Đó
        là nơi mà tư duy kiến trúc thực sự tỏa sáng
        -- \textbf{Yegor Bugayenko}
    \end{block}
\end{frame}

\begin{frame}{Tính không trùng lặp}
    \begin{itemize}
        \item Khi có \ul{\color{red}mã dư thừa} trong hệ thống, việc thay
        đổi hệ thống trở nên khó khăn. Các lỗi được tìm thấy
        phải được sửa trong mọi lần lặp lại. Tuy nhiên,
        vì mỗi lần lặp lại hơi khác nhau, nên việc sửa
        lỗi không phải lúc nào cũng giống nhau -- \textbf{Robert Martin}
    \end{itemize}
\end{frame}

\begin{frame}{Tính mô-đun hóa}
    \begin{itemize}
        \item Một thiết kế là \ul{\color{red}bất động} nếu nó
        chứa các thành phần
        được dùng trong các hệ thống khác, nhưng nỗ lực và
        rủi ro để tách các phần đó ra khỏi hệ thống ban đầu
        là quá lớn -- \textbf{Robert Martin}
        \item Một thiết kế là \ul{\color{red}cứng} nếu một
        thay đổi đơn lẻ gây ra một loạt các thay đổi tiếp
        theo trong các module phụ thuộc.
        Càng nhiều module phải thay đổi,
        thiết kế càng \ul{cứng} -- \textbf{Robert Martin}
    \end{itemize}
\end{frame}

\begin{frame}{Tài liệu tham khảo}
    \begin{columns}
        \column{0.5\textwidth}
        \begin{figure}
            \centering
            \includegraphics[width=0.4\textwidth]{../images/books/rup}
            \caption{Per Kroll và cộng sự, \emph{A Practitioner's Guide to the RUP}}
        \end{figure}
        \column{0.5\textwidth}
        \begin{figure}
            \centering
            \includegraphics[width=0.4\textwidth]{../images/books/agile-martin}
            \caption{Robert Martin,
            \emph{Agile Software Development. Principles, Patterns, and Practices}}
        \end{figure}
    \end{columns}
\end{frame}

\begin{frame}{Tài liệu tham khảo}
    \begin{columns}
        \column{0.5\textwidth}
        \begin{figure}
            \centering
            \includegraphics[width=0.4\textwidth]{../images/books/clean-architecture}
            \caption{Robert Martin, \emph{Clean Architecture: A Craftsman's Guide to Software Structure and Design}}
        \end{figure}
        \column{0.5\textwidth}
        \begin{figure}
            \centering
            \includegraphics[width=0.4\textwidth]{../images/books/code-complete}
            \caption{Steve McConnell, \emph{Code Complete}}
        \end{figure}
    \end{columns}
\end{frame}

\begin{frame}{Tài liệu tham khảo}
    \begin{columns}
        \column{0.5\textwidth}
        \begin{figure}
            \centering
            \includegraphics[width=0.4\textwidth]{../images/books/brooks}
            \caption{Frederick Brooks Jr., \emph{Mythical Man-Month, The: Essays on Software Engineering}}
        \end{figure}
        \column{0.5\textwidth}
        \begin{figure}
            \centering
            \includegraphics[width=0.4\textwidth]{../images/books/pragmatic}
            \caption{David Thomas và cộng sự, \emph{The Pragmatic Programmer: Your Journey To Mastery}}
        \end{figure}
    \end{columns}
\end{frame}

\end{document}
