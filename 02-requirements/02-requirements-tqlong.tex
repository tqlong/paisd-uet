\documentclass{beamer}

% Vietnamese support
\usepackage{fontenc}

\newcommand{\ul}[1]{\emph{#1}}

% Beamer theme
\usetheme{Berlin}

\title{Thực hành phát triển hệ thống Trí tuệ nhân tạo\\
Kỹ nghệ yêu cầu
}
\author{Trần Quốc Long}
\institute{Trường ĐH Công nghệ, ĐHQG Hà Nội}
\date{\today}

\begin{document}

\begin{frame}
    \titlepage
\end{frame}

\begin{frame}{Mục lục}
    \tableofcontents
\end{frame}

\section{Giới thiệu}

\begin{frame}{Giới thiệu}
    \begin{columns}
        \column{0.3\textwidth}
        \begin{figure}
            \centering
            \includegraphics[width=0.8\textwidth]{../images/books/agile-martin}
        \end{figure}
        \column{0.7\textwidth}
            \only<1>{
                \begin{block}{Yêu cầu}
                    If you are lucky, you start a project with a clear picture
                    of what you want the system to be. The design of the
                    system is a vital image in your mind. If you are luckier
                    still, the clarity of that design makes it to the first
                    release.\\
                    \textbf{Robert Martin}, \emph{Agile Software Development. Principles, Patterns,
                    and Practices}
                \end{block}
            }
            \only<2>{
                \begin{block}{Yêu cầu}
                    Nếu bạn may mắn, bạn bắt đầu một dự án với một cái nhìn rõ ràng
                    về hệ thống mong muốn. Thiết kế của
                    hệ thống là một hình ảnh quan trọng trong tâm trí bạn.
                    Nếu bạn may mắn hơn
                    nữa, sự rõ ràng của thiết kế vẫn còn nguyên ở phiên
                    bản phần mềm đầu tiên.\\
                    \textbf{Robert Martin}, \emph{Agile Software Development. Principles, Patterns,
                    and Practices}
                \end{block}
            }
        \end{columns}
\end{frame}

\begin{frame}{Giới thiệu}
    \begin{columns}
        \column{0.3\textwidth}
        \begin{figure}
            \centering
            \includegraphics[width=0.8\textwidth]{../images/books/code-ahead}
        \end{figure}
        \column{0.7\textwidth}
            \only<1>{
                \begin{block}{Yêu cầu}
                    Trách nhiệm của một lập trình viên là
                    đảm bảo rằng mọi nhiệm vụ đều có
                    \ul{ranh giới (yêu cầu)} rõ ràng.\\
                    \textbf{Yegor Bugayenko}, \emph{Code Ahead}
                \end{block}
            }
        \end{columns}
\end{frame}

\section{Ca sử dụng (Use Cases, Functional Requirements)}

\begin{frame}[label=interiors]{Ca sử dụng (Use Cases)}
    \begin{columns}
        \column{0.3\textwidth}
        \begin{figure}
            \centering
            \includegraphics[width=0.7\textwidth]{../images/faces/ivar-jacobson}
            \caption{Ivar Jacobson}
        \end{figure}
        \column{0.3\textwidth}
        \begin{figure}
            \centering
            \includegraphics[width=0.7\textwidth]{../images/faces/grady-booch}
            \caption{Grady Booch}
        \end{figure}
        \column{0.3\textwidth}
        \begin{figure}
            \centering
            \includegraphics[width=0.7\textwidth]{../images/faces/james-rumbaugh}
            \caption{James Rumbaugh}
        \end{figure}
    \end{columns}
\end{frame}

\begin{frame}{Ca sử dụng (Use Cases)}
    \begin{columns}
        \column{0.3\textwidth}
        \begin{figure}
            \centering
            \includegraphics[width=0.8\textwidth]{../images/books/uml-book}
        \end{figure}
        \column{0.7\textwidth}
            \only<1>{
                \begin{block}{Yêu cầu}
                    Không có tiêu chuẩn để viết một ca sử dụng
                    và mỗi trường hợp có thể dùng các định dạng khác nhau.\\
                    \textbf{Martin Fowler}, \emph{UML Distilled}
                \end{block}
            }
        \end{columns}
\end{frame}

\begin{frame}{Ví dụ về ca sử dụng (Who does What)}
    \begin{block}{}
        \begin{itemize}
            \item Ca sử dụng: Tạo mã QR
            \item Tác nhân: Người dùng
            \item Luồng cơ bản:
            \begin{enumerate}
                \item Người dùng nhập URL vào ô HTML.
                \item Hệ thống tạo hình ảnh PNG của mã QR.
                \item Người dùng tải hình ảnh qua HTTP.
            \end{enumerate}
            \item Mở rộng:
            \begin{enumerate}
                \item Người dùng nhập URL hỏng.
                \begin{enumerate}
                    \item Hệ thống hiển thị hộp thoại thông báo lỗi.
                    \item Người dùng xác nhận.
                \end{enumerate}
                \item Người dùng hủy tải về.
                \begin{enumerate}
                    \item Hệ thống dừng gửi dữ liệu qua HTTP.
                \end{enumerate}
            \end{enumerate}
        \end{itemize}
    \end{block}
\end{frame}

\begin{frame}{Biểu đồ ca sử dụng (Use Case Diagram)}
    \begin{figure}
        \centering
        \includegraphics[width=0.5\textwidth]{uc-diagram}
    \end{figure}
\end{frame}

% \section{Phân tích ca sử dụng}

\begin{frame}{Phân tích điểm chức năng (FPA)}
    \begin{block}{}
        \begin{itemize}
            \item Phân tích điểm chức năng (FPA) được phát triển
            bởi Allan Albrecht vào cuối những năm 1970 tại IBM.
            \item International Function Point Users Group (IFPUG).
            \item Được quy định bởi tiêu chuẩn ISO 20296, ISO/IEC 14143.
            \item Chia nhỏ yêu cầu thành bốn loại di chuyển dữ liệu: Entry (E), Exit (X), Read (R), Write (W).
        \end{itemize}
    \end{block}
    \Large Dùng để ước lượng khối lượng công việc (Function Points).
\end{frame}

%% TODO use case points

%% Traceability Matrix
\begin{frame}{Ma trận theo dõi (Traceability Matrix)}
    \begin{block}{}
        \begin{quote}
            Ma trận theo dõi là một công cụ quản lý dự án
            để theo dõi quan hệ giữa các yêu cầu và các yếu tố
            thiết kế của hệ thống.\\
            %% Use Cases, Non-Functional Requirements, Test Cases, Classes, Packages, Servers, Containers, Components, Nodes, etc.
            \small Ca sử dụng, yêu cầu phi chức năng, ca kiểm thử, lớp, gói, máy chủ, container, thành phần, node, v.v.
        \end{quote}
    \end{block}
    \begin{figure}
        \centering
        \includegraphics[width=0.5\textwidth]{matrix}
    \end{figure}
\end{frame}

\begin{frame}{Kiểm tra (verification) and Thẩm định (validation)}
    \begin{block}{Kiểm tra (Do it right?)}
        Kiểm tra là quá trình đảm bảo rằng phần mềm được
        xây dựng đúng theo các yêu cầu và đặc tả kỹ thuật.
    \end{block}
    \begin{block}{Thẩm định (Do the right thing?)}
        Thẩm định là quá trình đảm bảo rằng các yêu cầu và
        đặc tả kỹ thuật của phần mềm đáp ứng đúng nhu cầu và
        mong đợi của người dùng.
    \end{block}
\end{frame}

\section{Yêu cầu phi chức năng (Non-Functional)}

\begin{frame}{Yêu cầu phi chức năng (Non-Functional Requirements)}
    \begin{block}{}
        \begin{itemize}
            \item Hiệu suất (Performance, Capacity)
            \item Sẵn sàng (Availability)
            \item Bảo mật (Security)
            \item Khả năng phục hồi (Recoverability)
            \item Dễ sử dụng (Usability)
            \item Bảo trì (Maintainability)
            \item Di động (Mobile)
            \item Đáng tin cậy (Reliability)
            \item Tích hợp (Integration)
            \item V.v.
        \end{itemize}
    \end{block}
\end{frame}

\begin{frame}{Tính sẵn sàng}
    \begin{block}{}
        \begin{itemize}
            \item Sẵn sàng (Availability) là tỷ lệ thời gian hệ thống
            hoạt động so với tổng thời gian.
            \[A = \dfrac{E_\text{up}}{E_\text{down} + E_\text{up}}\]
        \end{itemize}
    \end{block}
\end{frame}

%% capacity
\begin{frame}{Hiệu suất}
    \begin{block}{}
        \begin{itemize}
            \item Hiệu suất (Capacity) là số lần giao tiếp mà hệ thống
            có thể xử lý trong một đơn vị thời gian.
            \[CPS = \text{Clicks Per Second}\]
        \end{itemize}
    \end{block}
\end{frame}

%% recovery time objective
\begin{frame}{Khả năng phục hồi}
    \begin{block}{}
        \begin{itemize}
            \item Khả năng phục hồi (Recovery) là thời gian cần thiết
            để hệ thống phục hồi sau một sự cố.
            \[RTO = \text{Recovery Time Objective}\]
        \end{itemize}
    \end{block}
\end{frame}

%% maintainability
\begin{frame}{Bảo trì}
    \begin{block}{}
        \begin{itemize}
            \item Bảo trì (Maintainability) là thời gian cần thiết
            để sửa một lỗi trong hệ thống.
            \[MTTR = \text{Mean Time To Repair}\]
        \end{itemize}
    \end{block}
\end{frame}

%% good and bad NFRs
\begin{frame}{Yêu cầu phi chức năng}
    \begin{columns}
    \column{0.5\textwidth}
        {\Huge \color{red}Không tốt}
        \begin{itemize}
            \item Phần mềm phải nhanh.
            \item Phải dễ bảo trì.
            \item Giao diện người dùng phải hấp dẫn.
        \end{itemize}
    \column{0.5\textwidth}
        {\Huge \color{blue}Tốt}
        \begin{itemize}
            \item Được cài đặt trên máy chủ kiểm thử (xem Phụ lục A),
            phần mềm phải phản hồi trong vòng 20ms
            cho các yêu cầu trong UC1--UC7.
            \item Thời gian tối đa để sửa một lỗi
            không quá hai giờ.
            \item Ít nhất 80\% người dùng beta phải xác nhận
            ẩn danh rằng giao diện người dùng đủ hấp dẫn.
        \end{itemize}
    \end{columns}
\end{frame}

%% các lỗi chính trong mô tả yêu cầu
\begin{frame}{Các lỗi chính trong mô tả yêu cầu}
    \begin{block}{}
        \begin{itemize}
            \item Không có từ điển thuật ngữ hoặc viết từ điển tồi.
            \item Dùng câu hỏi, thảo luận, ý kiến.
            \item Lẫn lộn yêu cầu chức năng và phi chức năng.
            \item Lẫn lộn yêu cầu và thông tin bổ sung.
            \item Yêu cầu phi chức năng không đo được.
            \item Đưa kèm hướng dẫn cài đặt, triển khai.
            \item Thiếu góc nhìn của người dùng.
            \item Nhiễu.
            \item Không rõ ràng (Who does What).
        \end{itemize}
    \end{block}
\end{frame}

\section{Ước lượng}

\begin{frame}{Ước lượng}
    \begin{columns}
        \column{0.3\textwidth}
        \begin{figure}
            \centering
            \includegraphics[width=0.8\textwidth]{../images/books/boehm}
        \end{figure}
        \column{0.7\textwidth}
            \only<1>{
                \begin{block}{}
                    Trong các quyết định thiết kế,
                    yếu tố quan trọng và khó
                    khăn nhất là ước tính
                    chi phí của một dự án phần mềm.\\
                    \textbf{Barry W. Boehm}, \emph{Software Engineering Economics}
                \end{block}
            }
    \end{columns}
\end{frame}

\begin{frame}{Ước lượng}
    \begin{block}{}
        \begin{enumerate}
            \item Ước lượng kích thước theo kilo-line-of-code
            \[K = FP \times \text{hệ số chuyển đổi}\]
            \item Tìm hệ số điều chỉnh năng lực ($F$).
            \item Tìm các hệ số $a$, $b$, $c$ và $d$ từ mô hình COCOMO II.
            \item Tính kích thước theo người-tháng (MM - man-month):
            \[E = a \times K^b \times F\]
            \item Tính thời gian theo tháng (D):
            \[D = c \times E^d\]
        \end{enumerate}
    \end{block}
\end{frame}

%% hệ số chuyển đổi
\begin{frame}{Hệ số chuyển đổi}
    \begin{block}{}
        \begin{itemize}
            \item Hệ số chuyển đổi (Conversion Factor) là tỷ lệ
            giữa kích thước ước lượng và kích thước thực tế.
            \[KLOC = FP \times \text{hệ số chuyển đổi}\]
            \item Hệ số chuyển đổi theo ngôn ngữ (KLOC/FP)
            \begin{itemize}
                \item Java: 0.4
                \item C\#: 0.5
                \item C: 0.6
                \item Python: 0.2
                \item COBOL: 0.8
            \end{itemize}
        \end{itemize}
    \end{block}
\end{frame}

%% hệ số điều chỉnh năng lực
\begin{frame}{Hệ số điều chỉnh năng lực}
    \begin{block}{}
        \begin{itemize}
            \item Hệ số điều chỉnh năng lực (Effort Adjustment Factor) là
            hệ số điều chỉnh dựa trên một số yếu tố ảnh hưởng
            đến khả năng thực hiện của dự án.
            \[F = \prod_{i=1}^n \text{hệ số hiệu chỉnh của yếu tố}_i\]
            \item Mô hình COCOMO II có 17 yếu tố ảnh hưởng.
        \end{itemize}
    \end{block}
\end{frame}

%% các hệ số a, b, c, d
\begin{frame}{Các hệ số $a$, $b$, $c$, $d$}
    \begin{block}{}
        \begin{itemize}
            \item Các hệ số $a$, $b$, $c$, $d$ được xác định từ
            mô hình COCOMO II.
            \item Các hệ số này phụ thuộc vào các pha của dự án
            (Early Design, Post-Architecture, Reuse Model).
            \item Các hệ số này được xác định từ bảng.
        \end{itemize}
    \end{block}
\end{frame}

%% cone of uncertainty
\begin{frame}{Nón không chắc chắn (Cone of Uncertainty)}
    \begin{figure}
        \centering
        \includegraphics[width=0.6\textwidth]{cone}
    \end{figure}
    \Large\color{red} "Phần mềm này sẽ mất bao nhiêu tiền?"\\
    "Việc phát triển sẽ kéo dài và tiêu hết tiền của bạn".
\end{frame}

%% books
\begin{frame}{Tài liệu tham khảo}
    \begin{columns}
        \column{0.5\textwidth}
        \begin{figure}
            \centering
            \includegraphics[width=0.4\textwidth]{../images/books/wiegers}
            \caption{Karl Wiegers, \emph{Software Requirements}}
        \end{figure}
        \column{0.5\textwidth}
        \begin{figure}
            \centering
            \includegraphics[width=0.4\textwidth]{../images/books/wiegers2}
            \caption{Karl Wiegers, \emph{More About Software Requirements}}
        \end{figure}
    \end{columns}
\end{frame}

\begin{frame}{Tài liệu tham khảo}
    \begin{columns}
        \column{0.5\textwidth}
        \begin{figure}
            \centering
            \includegraphics[width=0.4\textwidth]{../images/books/cockburn}
            \caption{Alistair Cockburn, \emph{Writing Effective Use Cases}}
        \end{figure}
        \column{0.5\textwidth}
        \begin{figure}
            \centering
            \includegraphics[width=0.4\textwidth]{../images/books/mcconnell}
            \caption{Steve McConnell, \emph{Software Estimation: Demystifying the Black Art}}
        \end{figure}
    \end{columns}
\end{frame}

\end{document}
