\documentclass{beamer}

% Vietnamese support
\usepackage{fontenc}

\newcommand{\ul}[1]{\emph{#1}}

% Beamer theme
\usetheme{Berlin}

\title{Thực hành phát triển hệ thống Trí tuệ nhân tạo\\
Tài liệu phần mềm
}
\author{Trần Quốc Long}
\institute{Trường ĐH Công nghệ, ĐHQG Hà Nội}
\date{\today}

\begin{document}

\begin{frame}
    \titlepage
\end{frame}

\begin{frame}{Mục lục}
    \tableofcontents
\end{frame}

\section{Giới thiệu}

\begin{frame}{Thiết kế}
    \begin{columns}
        \column{0.3\textwidth}
        \begin{figure}
            \centering
            \includegraphics[width=0.8\textwidth]{../images/books/freeman}
        \end{figure}
        \column{0.7\textwidth}
            \only<1>{
                \begin{block}{Định nghĩa}
                    Design encompasses all the activities involved in 
                    \ul{conceptualizing}, \ul{framing}, \ul{implementing}, \ul{commissioning}, 
                    and ultimately \ul{modifying} complex systems—not just the activity
                    following requirements specification and before programming, 
                    as it might be translated from a stylized software engineering
                    process.\\
                    \textbf{Peter Freeman and David Hart, \emph{Communications of the ACM}, vol. 47, no. 8, 2004}
                \end{block}
            }
            \only<2>{
                \begin{block}{Định nghĩa}
                    Thiết kế bao gồm tất cả các hoạt động liên quan đến 
                    \ul{khái niệm hóa}, \ul{định hình}, \ul{triển khai}, \ul{đưa vào vận hành} 
                    và cuối cùng là \ul{điều chỉnh} các hệ thống phức tạp—không chỉ là hoạt động
                    diễn ra sau khi đặc tả yêu cầu và trước khi lập trình, 
                    như cách tiếp cận trong một quy trình kỹ nghệ phần mềm truyền thống.\\
                    \textbf{Peter Freeman và David Hart, \emph{Communications of the ACM}, tập 47, số 8, 2004}
                \end{block}
            }
        \end{columns}
\end{frame}

\begin{frame}[label=interiors]{So sánh: Nội thất và Phần mềm}
    \begin{columns}
        \column{0.5\textwidth}
        \begin{figure}
            \centering
            \includegraphics[width=0.5\textwidth]{interior}
            \caption{Nội thất}
        \end{figure}
        \begin{figure}
            \centering
            \includegraphics[width=0.5\textwidth]{floor-plan}
            \caption{Thiết kế Nội thất}
        \end{figure}
        \column{0.5\textwidth}
        \begin{figure}
            \centering
            \includegraphics[width=0.5\textwidth]{docker-logo}
            \caption{Phần mềm}
        \end{figure}
        \begin{figure}
            \centering
            \includegraphics[width=0.5\textwidth]{docker-uml}
            \caption{Thiết kế Phần mềm}
        \end{figure}
    \end{columns}
\end{frame}

\begin{frame}{Giải thích \& thiết kế}
    \begin{figure}
        \centering
        \includegraphics[width=0.3\textwidth]{floor-plan}
        \caption{ {\textcolor{red}{Làm thế nào để giải thích?} --> Tiêu chuẩn} }
    \end{figure}
    \begin{figure}
        \centering
        \includegraphics[width=0.3\textwidth]{interior}
        \caption{ {\textcolor{blue}{Làm thế nào để thiết kế?} --> Mẫu thiết kế} }
    \end{figure}
\end{frame}

\section{Tài liệu mô tả thiết kế phần mềm}

\begin{frame}[label=sdd]{Tài liệu}
    \begin{block}{Tài liệu}
        % big font
        \Huge
        Một tài liệu tốt là \textcolor{red}{\ul{tiền đề}} cho một thiết kế tốt.
    \end{block}
\end{frame}

\begin{frame}{Mô tả thiết kế phần mềm (SDD) theo IEEE 1016}
    \begin{columns}
        \column{0.3\textwidth}
        \begin{figure}
            \centering
            \includegraphics[width=0.8\textwidth]{../images/books/ieee-1016}
        \end{figure}
        \column{0.7\textwidth}
        \begin{block}{Định nghĩa}
            Một SDD (Software Design Description) là biểu diễn của
            thiết kế phần mềm nhằm ghi lại thông tin thiết kế
            và truyền tải thông tin thiết kế đó đến
            các bên liên quan.
            Tiêu chuẩn này được sử dụng trong
            các tình huống thiết kế yêu cầu phải chuẩn
            bị SDD rõ ràng.\\
            \textbf{IEEE 1016-2009: Tiêu chuẩn IEEE về Công nghệ Thông tin---Thiết kế Hệ thống---Mô tả Thiết kế Phần mềm}
        \end{block}
        \end{columns}
\end{frame}

\begin{frame}{Các tài liệu trong SDD}
    \begin{enumerate}
        \item Từ điển thuật ngữ (Glossary)
        \item Ngôn ngữ thiết kế (Languages)
        \item Các bên liên quan (Stakeholders)
        \item Các mối quan tâm (Concerns)
        \\Yêu cầu: Chức năng (Functional) \& Phi chức năng (Non-functional)
        \item Các lát cắt (Viewpoints)
        \item Các thành phần (Elements)
        \item Lý do cho các quyết định thiết kế (Rationale)
    \end{enumerate}
\end{frame}

\begin{frame}{Từ điển thuật ngữ}
    \begin{block}{Từ điển thuật ngữ}
        \begin{itemize}
            \item \ul{Yêu cầu (request)} là gói dữ liệu gửi từ \ul{máy khách} đến \ul{máy chủ}.
            \item \ul{Máy khách (client)} là máy tính có trình duyệt web.
            \item \ul{Máy chủ (server)} là máy tính có phần mềm cài đặt.
        \end{itemize}
    \end{block}
    \onslide<2>{
    \Huge
    \textcolor{red}{Nếu bạn không hiểu, đó là lỗi của tôi!}
    }
\end{frame}

\begin{frame}{Ngôn ngữ thiết kế}
    \begin{columns}
        \column{0.5\textwidth}
        \begin{figure}
            \centering
            \includegraphics[width=0.5\textwidth]{bad-diagram}
            \caption{Không tốt}
        \end{figure}
        \column{0.5\textwidth}
        \begin{figure}
            \centering
            \includegraphics[width=0.7\textwidth]{good-uml}
            \caption{Tốt (UML)}
        \end{figure}
    \end{columns}
\end{frame}

\begin{frame}{Các bên liên quan}
    \begin{block}{Các bên liên quan}
        \begin{quote}
            Xác định các bên liên quan là quá trình
            xác định những người, nhóm hoặc tổ chức
            có thể ảnh hưởng hoặc bị ảnh hưởng bởi
            quyết định, hoạt động hoặc kết quả của dự án.
        \end{quote}
    \end{block}
\end{frame}

\begin{frame}{Các mối quan tâm}
    Yêu cầu
    \begin{itemize}
        \item Chức năng (Functional)
        \item Phi chức năng (Non-functional)
    \end{itemize}
\end{frame}

\begin{frame}{Các lát cắt}
    \begin{block}{}
        \begin{quote}
            Một lát cắt là một cách nhìn vào hệ thống
            từ một góc độ cụ thể, mô tả một số khía cạnh
            của hệ thống và cung cấp thông tin cần thiết
            cho một số bên liên quan.
        \end{quote}
    \end{block}
    \begin{figure}
        \centering
        \includegraphics[width=0.5\textwidth]{viewpoint}
    \end{figure}
\end{frame}

\begin{frame}{Các thành phần}
    \begin{block}{Ví dụ}
        \begin{itemize}
            \item Mô-đun
            \item Hành vi
            \item Dữ liệu
            \item Giao diện
        \end{itemize}
    \end{block}
    \begin{figure}
        \centering
        \includegraphics[width=0.5\textwidth]{element}
    \end{figure}
\end{frame}

\begin{frame}{Lý do cho các quyết định thiết kế}
    \only<1>{
        \begin{block}{Lý do}
            \begin{itemize}
                \item Lý do chọn mô-đun A thay vì mô-đun B
                \item Lý do chọn giao diện A thay vì giao diện B
                \item Lý do chọn cơ sở dữ liệu A thay vì cơ sở dữ liệu B
                \item Lý do chọn công nghệ A thay vì công nghệ B
                \item Lý do chọn mẫu thiết kế A thay vì mẫu thiết kế B
            \end{itemize}
        \end{block}
    }
    \only<2>{
    \begin{block}{Công cụ}
        \begin{itemize}
            \item Quyết định đa tiêu chí (Multi-Criteria Decision Making (MCDM))
            \item Phân tích lựa chọn thiết kế (Architecture Trade-off Analysis Method (ATAM))
            \item Bảng quyết định (Decision Table)
            \item Phân tích nhiều yếu tố (Multi Factor Analysis)
            \item Ma trận quyết định (Decision Matrix)
        \end{itemize}
    \end{block}
    \Large \textcolor{red}{
        Không cần tin tôi, hãy tin quyết định của tôi !!!
    }
    }
\end{frame}

\section{Công cụ viết tài liệu}

\begin{frame}{Công cụ viết tài liệu}
    \begin{block}{Công cụ}
        \begin{itemize}
            \item \LaTeX
            \item Markdown (Notion, GitHub)
            \item Visual Paradigm
            \item UML
            \item UI Mockups
        \end{itemize}
    \end{block}
\end{frame}

\end{document}
