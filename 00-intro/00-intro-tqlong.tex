\documentclass{beamer}

% Vietnamese support
\usepackage{fontenc}

% Beamer theme
\usetheme{Berlin}

\title{Thực hành phát triển hệ thống Trí tuệ nhân tạo}
\author{Trần Quốc Long}
\institute{Trường ĐH Công nghệ, ĐHQG Hà Nội}
\date{\today}

\begin{document}

\begin{frame}
    \titlepage
\end{frame}

\begin{frame}{Mục lục}
    \tableofcontents
\end{frame}

\section{Giới thiệu học phần}

\begin{frame}{Giới thiệu chung}
    Các bài giảng trong học phần này:
    \begin{itemize}
        \item Chỉ mang tính tổng quan, không đi sâu vào chi tiết
        \item Không thay thế Wikipedia hoặc sách
        \item Không phải lúc nào cũng chính xác
        \item Được trình bày lần đầu tiên
        \item Chi tiết nằm trong các ví dụ, bài tập và dự án kèm theo
    \end{itemize}
\end{frame}

\begin{frame}{Các bài giảng}
    \begin{enumerate}
        \item README vs. IEEE
        \\ Kỹ nghệ yêu cầu (Requirements Engineering)
        \\ RUP vs. Agile/XP
        \item OOAD + Nguyên tắc và Mùi thiết kế
        \\ Tư duy hướng đối tượng (Object Thinking) + DDD
        \item Mẫu thiết kế (Patterns), Phản mẫu (Anti-Patterns)\\Cải tiến mã (Refactoring)
        \item XML và JSON
        \\ Ngôn ngữ mô hình hóa thống nhất (UML)
    \end{enumerate}        
\end{frame}


\begin{frame}{Các bài giảng}
    \begin{enumerate}
        \setcounter{enumi}{4}
        \item Thiết kế dữ liệu
        \item Thiết kế cho phân phối liên tục
        \\Microservices, APIs, RESTful
        \\Thiết kế Serverless trên Cloud
        \item Phát triển dựa trên Kiểm thử (TDD)
        \\Phản mẫu kiểm thử (Test Anti-Patterns)
        \item Liên kết \& Kết hợp và các chỉ số khác
        \\Tương lai của thiết kế phần mềm
    \end{enumerate}        
\end{frame}

\section{Chính sách và quy định}

\begin{frame}{Chính sách và quy định}
    \begin{itemize}
        \item Điểm danh: thông qua quiz trên portal, theo quy chế của trường
        \item Điểm thành phần (40\%): bài tập nửa đầu học kì (20\%), thi giữa kì (20\%) 
        \item Điểm cuối kì (60\%): bài tập nửa sau học kì (20\%), dự án/vấn đáp (40\%)
    \end{itemize}
\end{frame}

\section{Tài nguyên}

\begin{frame}{Tài nguyên}
    Sách tham khảo:
    \begin{itemize}
        \item \textit{Elegant Objects, Code Ahead} (Yegor Bugayenko)
        \item \textit{Refactoring, UML} (Martin Fowler)
        \item \textit{Clean Code, Agile} (Robert C. Martin)
    \end{itemize}
\end{frame}

% \section{Nội dung chính}

% \begin{frame}{Chủ đề 1}
%     \begin{block}{Định nghĩa}
%         Đây là nội dung của bài giảng.
%     \end{block}
% \end{frame}

% \begin{frame}{Chủ đề 2}
%     \begin{example}
%         Đây là một ví dụ minh họa.
%     \end{example}
% \end{frame}

% \section{Kết luận}

% \begin{frame}{Tóm tắt}
%     \begin{enumerate}
%         \item Ý chính thứ nhất
%         \item Ý chính thứ hai
%         \item Ý chính thứ ba
%     \end{enumerate}
% \end{frame}

% \begin{frame}{Câu hỏi và thảo luận}
%     \centering
%     \Huge Câu hỏi?
% \end{frame}

\end{document}
