\documentclass{beamer}

% Vietnamese support
\usepackage{fontenc}
\usepackage{qrcode}

\newcommand{\ul}[1]{\emph{#1}}

% Beamer theme
\usetheme{Berlin}

\title{Thực hành phát triển hệ thống Trí tuệ nhân tạo\\
Quản lý dữ liệu và pipeline xử lý dữ liệu
}
\author{Trần Quốc Long}
\institute{Trường ĐH Công nghệ, ĐHQG Hà Nội}
\date{\today}

\begin{document}
\begin{frame}
    \titlepage
\end{frame}

\begin{frame}{Mục lục}
    \tableofcontents
\end{frame}

\section{Thu thập dữ liệu}

\begin{frame}{Thu thập dữ liệu}
    \begin{columns}
        \column{0.1\textwidth}
        \begin{figure}
            \centering
            % \includegraphics[width=0.8\textwidth]{../images/data-collection}
        \end{figure}
        \column{0.9\textwidth}
        \begin{itemize}
            \item Các phương pháp thu thập dữ liệu:
            \begin{itemize}
                \item Nguồn công khai: Kaggle, UCI Machine Learning Repository, \dots
                \item API: Twitter API, Google Maps API, \dots
                \item Web scraping: BeautifulSoup, Scrapy, \dots
            \end{itemize}
            \item Đảm bảo dữ liệu có tính hợp lệ và không bị nhiễu:
            \begin{itemize}
                \item Kiểm tra tính hợp lệ của dữ liệu
                \item Loại bỏ dữ liệu nhiễu
                \item Đảm bảo dữ liệu phù hợp với bài toán
            \end{itemize}
        \end{itemize}
    \end{columns}
\end{frame}

\begin{frame}{Dữ liệu hình ảnh}
    \begin{itemize}
        \item \textbf{Đặc điểm}: Dữ liệu thị giác, đa định dạng (JPEG, PNG, \dots)
        \item \textbf{Thách thức}:
        \begin{itemize}
            \item Kích thước tệp lớn
            \item Cần chú thích (ví dụ: gắn nhãn cho học có giám sát)
            \item Khả năng thiên vị trong tập dữ liệu hình ảnh
            \item Cân bằng tập dữ liệu cho các tác vụ phân loại
        \end{itemize}
        \item \textbf{Nguồn}: Tập dữ liệu công khai (ví dụ: ImageNet, COCO), web scraping, nội dung do người dùng tạo
        \item \textbf{Công cụ}: OpenCV, Pillow để xử lý; công cụ chú thích như LabelImg
        \item \textbf{Lưu ý}:
        \begin{itemize}
            \item Đảm bảo sự đa dạng trong tập dữ liệu để tránh thiên vị
            \item Xử lý các độ phân giải và tỷ lệ khung hình khác nhau
            \item Xem xét các kỹ thuật tăng cường dữ liệu
            \item Cân bằng tập dữ liệu để tránh thiên vị lớp
        \end{itemize}
    \end{itemize}
\end{frame}

\begin{frame}{Dữ liệu văn bản}
    \begin{itemize}
        \item \textbf{Đặc điểm}: Ngôn ngữ tự nhiên, đa ngôn ngữ
        \item \textbf{Thách thức}:
        \begin{itemize}
            % \item Xử lý dành riêng cho ngôn ngữ
            \item Xử lý các mã hóa khác nhau
            \item Xử lý văn bản không cấu trúc
            \item Xử lý các ngôn ngữ khác nhau %và có thể cần dịch thuật
        \end{itemize}
        \item \textbf{Nguồn}: Web scraping, API (ví dụ: Twitter), tập dữ liệu công khai (ví dụ: Wikipedia dumps)
        \item \textbf{Công cụ}: NLTK, spaCy cho các tác vụ NLP; BeautifulSoup cho web scraping
        \item \textbf{Lưu ý}:
        \begin{itemize}
            \item Tiền xử lý văn bản (tokenization, stemming, \dots)
            \item Xử lý các từ dừng (stop words)
            \item Dịch ngôn ngữ nếu cần
            \item Xử lý các phương ngữ hoặc biến thể ngôn ngữ
        \end{itemize}
    \end{itemize}
\end{frame}

\begin{frame}{Dữ liệu âm thanh}
    \begin{itemize}
        \item \textbf{Đặc điểm}: Dữ liệu chuỗi thời gian
        \item \textbf{Thách thức}:
        \begin{itemize}
            \item Giảm nhiễu
            \item Trích xuất đặc trưng
            \item Xử lý các định dạng âm thanh khác nhau
            \item Xử lý các giọng nói hoặc phương ngữ khác nhau
        \end{itemize}
        \item \textbf{Nguồn}: Tập dữ liệu công khai (ví dụ: LibriSpeech), ghi âm riêng, API
        \item \textbf{Công cụ}: Librosa, PyDub để xử lý âm thanh; thư viện nhận dạng giọng nói
        \item \textbf{Lưu ý}:
        \begin{itemize}
            \item Tốc độ lấy mẫu
            \item Chất lượng âm thanh
            \item Phiên âm cho dữ liệu giọng nói
            \item Đa dạng hóa tập dữ liệu để bao gồm các giọng nói khác nhau
        \end{itemize}
    \end{itemize}
\end{frame}

\begin{frame}{Dữ liệu chuỗi thời gian}
    \begin{itemize}
        \item \textbf{Đặc điểm}: Có tính xu hướng, hoặc mùa vụ
        \item \textbf{Thách thức}:
        \begin{itemize}
            \item Giá trị thiếu
            \item Lấy mẫu không đều
        \end{itemize}
        \item \textbf{Nguồn}: Cảm biến, dữ liệu tài chính, dữ liệu thời tiết
        \item \textbf{Công cụ}: Pandas, statsmodels để phân tích; Prophet để dự báo
        \item \textbf{Lưu ý}:
        \begin{itemize}
            \item Xử lý dữ liệu thiếu
            \item Kỹ thuật đặc trưng dựa trên thời gian
            \item Lựa chọn mô hình cho dự báo chuỗi thời gian
            \item Kiểm tra tính dừng
            \item Biến đổi dữ liệu nếu cần
        \end{itemize}
    \end{itemize}
\end{frame}

\begin{frame}{Dữ liệu có cấu trúc}
    \begin{itemize}
        \item \textbf{Đặc điểm}: Tổ chức thành bảng, thường có schema cho trước
        \item \textbf{Thách thức}:
        \begin{itemize}
            \item Làm sạch dữ liệu
            \item Xử lý biến phân loại
            \item Lựa chọn đặc trưng
            \item Xử lý phân bố không cần bằng
        \end{itemize}
        \item \textbf{Nguồn}: Cơ sở dữ liệu, tệp CSV, API cung cấp dữ liệu dạng bảng
        \item \textbf{Công cụ}: Pandas, SQL, Scikit-learn
        \item \textbf{Lưu ý}:
        \begin{itemize}
            \item Xử lý giá trị thiếu
            \item Mã hóa biến phân loại
            \item Chuẩn hóa hoặc chia tỷ lệ các đặc trưng
            \item Xử lý các lớp không cân bằng (ví dụ: oversampling, undersampling)
        \end{itemize}
    \end{itemize}
\end{frame}

\section{Xử lý dữ liệu}

\begin{frame}{Xử lý dữ liệu}
    \begin{columns}
        \column{0.1\textwidth}
        \begin{figure}
            \centering
            % \includegraphics[width=0.8\textwidth]{../images/data-processing}
        \end{figure}
        \column{0.9\textwidth}
        \begin{itemize}
            \item Làm sạch và chuẩn hóa dữ liệu:
            \begin{itemize}
                \item Loại bỏ dữ liệu trùng lặp
                \item Chuẩn hóa định dạng dữ liệu
                \item Chuyển đổi dữ liệu sang định dạng phù hợp
            \end{itemize}
            \item Xử lý dữ liệu mất giá trị và dữ liệu ngoại lai:
            \begin{itemize}
                \item Điền giá trị mất bằng trung bình, median, \dots
                \item Loại bỏ hoặc xử lý dữ liệu ngoại lai
                \item Sử dụng kỹ thuật interpolation
            \end{itemize}
            \item Chuẩn bị dữ liệu cho mô hình:
            \begin{itemize}
                \item Chia dữ liệu thành tập huấn luyện, kiểm thử, đánh giá
                \item Chọn lựa đặc trưng
                \item Biến đổi dữ liệu (scaling, normalization, \dots)
            \end{itemize}
        \end{itemize}
    \end{columns}
\end{frame}

\begin{frame}{Xử lý dữ liệu hình ảnh}
    \begin{itemize}
        \item Thay đổi kích thước hình ảnh thành kích thước tiêu chuẩn
        \item Chuẩn hóa giá trị pixel (ví dụ: chia cho 255)
        \item Tăng cường dữ liệu:
        \begin{itemize}
            \item Cắt, xoay, lật hình ảnh
            \item Thay đổi độ sáng, độ tương phản
        \end{itemize}
        \item Chuyển đổi sang thang độ xám nếu cần
        \item Áp dụng bộ lọc để giảm nhiễu
    \end{itemize}
\end{frame}

\begin{frame}{Xử lý dữ liệu văn bản}
    \begin{itemize}
        \item Tokenization: chia văn bản thành các từ hoặc câu
        \item Loại bỏ từ dừng (stop words)
        \item Stemming hoặc lemmatization để đưa từ về dạng gốc
        \item Chuyển đổi văn bản sang chữ thường
        \item Vector hóa văn bản:
        \begin{itemize}
            \item Sử dụng TF-IDF
            \item Sử dụng word embeddings (Word2Vec, GloVe)
        \end{itemize}
    \end{itemize}
\end{frame}

\begin{frame}{Xử lý dữ liệu âm thanh}
    \begin{itemize}
        \item Chuyển đổi sang định dạng âm thanh tiêu chuẩn (ví dụ: WAV)
        \item Resampling để có tần số lấy mẫu nhất quán
        \item Trích xuất đặc trưng:
        \begin{itemize}
            \item MFCCs (Mel-frequency cepstral coefficients)
            \item Spectrograms
        \end{itemize}
        \item Giảm nhiễu âm thanh
        \item Chuẩn hóa biên độ
    \end{itemize}
\end{frame}

\begin{frame}{Xử lý dữ liệu chuỗi thời gian}
    \begin{itemize}
        \item Xử lý giá trị thiếu:
        \begin{itemize}
            \item Nội suy (interpolation)
            \item Điền giá trị bằng giá trị trước đó (forward fill)
        \end{itemize}
        \item Xử lý lấy mẫu không đều:
        \begin{itemize}
            \item Resampling thành tần suất cố định
        \end{itemize}
        \item Loại bỏ xu hướng và tính mùa vụ nếu cần
        \item Tạo đặc trưng:
        \begin{itemize}
            \item Đặc trưng trễ (lag features)
            \item Thống kê cửa sổ trượt (rolling statistics)
        \end{itemize}
    \end{itemize}
\end{frame}

\begin{frame}{Xử lý dữ liệu có cấu trúc}
    \begin{itemize}
        \item Xử lý giá trị thiếu:
        \begin{itemize}
            \item Điền bằng giá trị trung bình, median, mode
            \item Sử dụng mô hình dự đoán để điền giá trị thiếu
        \end{itemize}
        \item Mã hóa biến phân loại:
        \begin{itemize}
            \item One-hot encoding
            \item Label encoding
        \end{itemize}
        \item Chuẩn hóa hoặc chia tỷ lệ các đặc trưng số:
        \begin{itemize}
            \item Min-max scaling
            \item Standardization (z-score)
        \end{itemize}
        \item Phát hiện và xử lý dữ liệu ngoại lai
    \end{itemize}
\end{frame}

\section{Quản lý pipeline xử lý dữ liệu}

\begin{frame}{Quản lý pipeline xử lý dữ liệu}
    \begin{columns}
        \column{0.1\textwidth}
        \begin{figure}
            \centering
            % \includegraphics[width=0.8\textwidth]{../images/pipeline}
        \end{figure}
        \column{0.9\textwidth}
        \begin{itemize}
            \item Sử dụng Data Version Control (DVC):
            \begin{itemize}
                \item Quản lý phiên bản dữ liệu
                \item Theo dõi thay đổi trong dữ liệu
                \item Chia sẻ dữ liệu giữa các thành viên
            \end{itemize}
            \item Khái niệm ETL trong MLOps:
            \begin{itemize}
                \item Extract: Thu thập dữ liệu từ các nguồn
                \item Transform: Xử lý dữ liệu cho mô hình
                \item Load: Lưu trữ dữ liệu đã xử lý
            \end{itemize}
        \end{itemize}
    \end{columns}
\end{frame}

\begin{frame}{Data Version Control (DVC) là gì?}
    \begin{itemize}
        \item \textbf{DVC}
        \begin{itemize}
            \item Một công cụ quản lý phiên bản dữ liệu.
            \item Tương tự Git nhưng dành cho dữ liệu và mô hình.
        \end{itemize}
        \item \textbf{Mục đích}:
        \begin{itemize}
            \item Theo dõi và quản lý các phiên bản dữ liệu.
            \item Đảm bảo tính nhất quán và khả năng tái lập.
        \end{itemize}
    \end{itemize}
\end{frame}

\begin{frame}{Tại sao cần DVC?}
    \begin{itemize}
        \item \textbf{Thách thức với dữ liệu lớn}:
        \begin{itemize}
            \item Dữ liệu lớn không thể quản lý bằng Git thông thường.
            \item Cần công cụ chuyên biệt để xử lý dữ liệu lớn.
        \end{itemize}
        \item \textbf{Lợi ích của DVC}:
        \begin{itemize}
            \item Quản lý phiên bản dữ liệu dễ dàng.
            \item Theo dõi thay đổi trong dữ liệu.
            \item Chia sẻ dữ liệu giữa các thành viên nhóm.
        \end{itemize}
    \end{itemize}
\end{frame}

\begin{frame}{Cách hoạt động của DVC}
    \begin{itemize}
        \item \textbf{Cơ chế hoạt động}:
        \begin{itemize}
            \item DVC lưu trữ dữ liệu trong bộ nhớ cache.
            \item Sử dụng file .dvc để theo dõi phiên bản.
            \item Liên kết với Git để quản lý mã và dữ liệu.
        \end{itemize}
        \item \textbf{Các lệnh cơ bản}:
        \begin{itemize}
            \item \texttt{dvc init}: Khởi tạo DVC.
            \item \texttt{dvc add}: Thêm dữ liệu vào DVC.
            \item \texttt{dvc push}: Đẩy dữ liệu lên remote.
        \end{itemize}
    \end{itemize}
\end{frame}

\begin{frame}{Lợi ích của DVC trong MLOps}
    \begin{itemize}
        \item \textbf{Tái lập thí nghiệm}: Dễ dàng khôi phục dữ liệu cũ.
        \item \textbf{Hợp tác nhóm}: Chia sẻ dữ liệu và mô hình dễ dàng.
        \item \textbf{Quản lý pipeline}: Tích hợp với pipeline xử lý dữ liệu.
        \item \textbf{Tiết kiệm không gian}: Không lưu trữ dữ liệu trong Git.
    \end{itemize}
\end{frame}

\begin{frame}{ETL là gì?}
    \begin{itemize}
        \item \textbf{ETL} là viết tắt của:
        \begin{itemize}
            \item \textbf{Extract}: Thu thập dữ liệu từ các nguồn khác nhau.
            \item \textbf{Transform}: Xử lý và chuyển đổi dữ liệu thành định dạng phù hợp.
            \item \textbf{Load}: Tải dữ liệu đã xử lý vào hệ thống đích.
        \end{itemize}
        \item \textbf{Mục đích}: Chuẩn bị dữ liệu cho các ứng dụng phân tích, báo cáo, hoặc học máy.
    \end{itemize}
\end{frame}

\begin{frame}{Extract: Thu thập dữ liệu}
    \begin{itemize}
        \item \textbf{Nguồn dữ liệu}:
        \begin{itemize}
            \item Cơ sở dữ liệu (SQL, NoSQL)
            \item Tệp tin (CSV, JSON, XML)
            \item API (REST, SOAP)
            \item Web scraping
        \end{itemize}
        \item \textbf{Phương pháp}:
        \begin{itemize}
            \item Batch processing: Thu thập dữ liệu theo lô.
            \item Real-time processing: Thu thập dữ liệu theo thời gian thực.
        \end{itemize}
    \end{itemize}
\end{frame}

\begin{frame}{Transform: Xử lý dữ liệu}
    \begin{itemize}
        \item \textbf{Kỹ thuật xử lý}:
        \begin{itemize}
            \item Làm sạch dữ liệu: Loại bỏ dữ liệu trùng lặp, xử lý giá trị thiếu.
            \item Chuẩn hóa: Đưa dữ liệu về định dạng thống nhất.
            \item Biến đổi: Tính toán các đặc trưng mới, mã hóa biến phân loại.
            \item Kết hợp: Ghép nối dữ liệu từ nhiều nguồn.
        \end{itemize}
        \item \textbf{Công cụ}: Python (Pandas, NumPy), SQL, Apache Spark.
    \end{itemize}
\end{frame}

\begin{frame}{Load: Tải dữ liệu}
    \begin{itemize}
        \item \textbf{Hệ thống đích}:
        \begin{itemize}
            \item Kho dữ liệu (Data Warehouse)
            \item Hồ dữ liệu (Data Lake)
            \item Cơ sở dữ liệu phân tích
            \item Hệ thống học máy
        \end{itemize}
        \item \textbf{Phương pháp}:
        \begin{itemize}
            \item Batch loading: Tải dữ liệu theo lô.
            \item Incremental loading: Tải dữ liệu mới hoặc cập nhật.
        \end{itemize}
    \end{itemize}
\end{frame}

\begin{frame}{ETL trong MLOps}
    \begin{itemize}
        \item \textbf{Vai trò}:
        \begin{itemize}
            \item Cung cấp dữ liệu chất lượng cao cho huấn luyện mô hình.
            \item Đảm bảo tính nhất quán và khả năng tái lập.
            \item Tự động hóa quy trình xử lý dữ liệu.
        \end{itemize}
        \item \textbf{Lợi ích}:
        \begin{itemize}
            \item Tiết kiệm thời gian và công sức.
            \item Giảm thiểu lỗi thủ công.
            \item Dễ dàng mở rộng cho các tập dữ liệu lớn.
        \end{itemize}
    \end{itemize}
\end{frame}

\begin{frame}{Ví dụ ETL và DVC cho dữ liệu hình ảnh}
    \begin{itemize}
        \item \textbf{Extract}: Thu thập hình ảnh từ Flickr API.
        \item \textbf{Transform}:
        \begin{itemize}
            \item Resize hình ảnh về kích thước 256x256.
            \item Chuẩn hóa giá trị pixel từ 0 đến 1.
            \item Áp dụng tăng cường dữ liệu: xoay, lật.
        \end{itemize}
        \item \textbf{Load}: Lưu trữ hình ảnh đã xử lý vào Amazon S3.
        \item \textbf{DVC}:
        \begin{itemize}
            \item Sử dụng \texttt{dvc add} để theo dõi dữ liệu hình ảnh.
            \item Commit thay đổi với Git và đẩy dữ liệu lên remote với \texttt{dvc push}.
            \item Quản lý các phiên bản khác nhau của dữ liệu hình ảnh.
        \end{itemize}
    \end{itemize}
\end{frame}

\end{document}