\documentclass{beamer}

% Vietnamese support
\usepackage{fontenc}
\usepackage{qrcode}

\newcommand{\ul}[1]{\emph{#1}}

% Beamer theme
\usetheme{Berlin}

\title{Thực hành phát triển hệ thống Trí tuệ nhân tạo\\
Tư duy hướng đối tượng (Object Thinking)
}
\author{Trần Quốc Long}
\institute{Trường ĐH Công nghệ, ĐHQG Hà Nội}
\date{\today}

\begin{document}

\begin{frame}
    \titlepage
\end{frame}

\begin{frame}{Mục lục}
    \tableofcontents
\end{frame}

\section{Triết lý của OOP}

\begin{frame}{OOP ???}
    \begin{columns}
        \column{0.3\textwidth}
        \begin{figure}
            \centering
            \includegraphics[width=0.8\textwidth]{../images/faces/edsger-dijkstra}
        \end{figure}
        \column{0.6\textwidth}
        Object-oriented programs are offered as alternatives to
        correct ones \\-- \textbf{Edsger W. Dijkstra} (1989)
    \end{columns}
\end{frame}

\begin{frame}{OOP ???}
    \begin{columns}
        \column{0.3\textwidth}
        \begin{figure}
            \centering
            \includegraphics[width=0.8\textwidth]{../images/faces/alan-kay}
        \end{figure}
        \column{0.6\textwidth}
        I invented the term \emph{object-oriented}, and I can tell
        you I did not have C++ in mind\\
        -- \textbf{Alan Kay} (1997)
    \end{columns}
\end{frame}

\begin{frame}{OOP ???}
    \begin{columns}
        \column{0.3\textwidth}
        \begin{figure}
            \centering
            \includegraphics[width=0.8\textwidth]{../images/faces/paul-graham}
        \end{figure}
        \column{0.6\textwidth}
        Object-oriented programming offers a sustainable way to write
        spaghetti code\\
        -- \textbf{Paul Graham} (2003)
    \end{columns}
\end{frame}

\begin{frame}{OOP ???}
    \begin{columns}
        \column{0.3\textwidth}
        \begin{figure}
            \centering
            \includegraphics[width=0.8\textwidth]{../images/faces/linus-torvalds}
        \end{figure}
        \column{0.6\textwidth}
        C++ is a horrible language. C++ leads to really, really bad
        design choices. ... idiotic object model crap.\\
        -- \textbf{Linus Torvalds} (2007)
    \end{columns}
\end{frame}

\begin{frame}{Lịch sử}
    \Large Thời kì GOTO
    \begin{figure}
        \centering
        \includegraphics[width=0.7\textwidth]{basic}
    \end{figure}
\end{frame}

\begin{frame}{Lịch sử}
    \Large Lập trình có cấu trúc
    \begin{columns}
        \column{0.5\textwidth}
        \begin{figure}
            \centering
            \includegraphics[width=0.9\textwidth]{pascal}
        \end{figure}
        \column{0.5\textwidth}
        \begin{figure}
            \centering
            \includegraphics[width=0.7\textwidth]{flowchart}
        \end{figure}
    \end{columns}
\end{frame}

\begin{frame}{Lịch sử}
    \Large Lập trình thủ tục
    \begin{figure}
        \centering
        \includegraphics[width=0.5\textwidth]{calls}
    \end{figure}
\end{frame}

\begin{frame}{Lịch sử}
    \Large Lập trình hướng đối tượng ... chưa tới!
    \begin{figure}
        \centering
        \includegraphics[width=0.5\textwidth]{oop-bad}
    \end{figure}
\end{frame}

\begin{frame}{Lịch sử}
    \Large Code OOP kiểu mì Ý!
    \begin{figure}
        \centering
        \includegraphics[width=0.8\textwidth]{rectangles1}
    \end{figure}
\end{frame}

\begin{frame}{Lịch sử}
    \Large Lập trình hướng đối tượng ... đúng cách!
    \begin{figure}
        \centering
        \includegraphics[width=0.5\textwidth]{oop-good}
    \end{figure}
\end{frame}

\begin{frame}{Lịch sử}
    \Large Code OOP kiểu tao nhã!
    \begin{figure}
        \centering
        \includegraphics[width=0.8\textwidth]{rectangles2}
    \end{figure}
\end{frame}

\section{Đối tượng là gì?}

\begin{frame}{C++}
    \begin{columns}
        \column{0.3\textwidth}
        \begin{figure}
            \centering
            \includegraphics[width=0.8\textwidth]{../images/books/stroustrup}
        \end{figure}
        \column{0.7\textwidth}
        An object is some memory that holds a value of some type.\\
        -- \textbf{Bjarne Stroustrup}, \emph{Programming Principles and Practice Using C++}
    \end{columns}
\end{frame}

\begin{frame}{Wikipedia}
    \begin{columns}
        \column{0.3\textwidth}
        \begin{figure}
            \centering
            \includegraphics[width=0.8\textwidth]{../images/books/wikipedia}
        \end{figure}
        \column{0.7\textwidth}
        Objects may contain data, in the form of fields, often known
        as attributes; and code, in the form of procedures,
        often known as methods.\\
        -- \emph{Wikipedia}
    \end{columns}
\end{frame}

\begin{frame}{Smalltalk}
    \begin{columns}
        \column{0.3\textwidth}
        \begin{figure}
            \centering
            \includegraphics[width=0.8\textwidth]{../images/books/smalltalk}
        \end{figure}
        \column{0.7\textwidth}
        An object consists of some private memory and a set of
        operations.\\
        -- \textbf{Adele Goldberg và cộng sự}, \emph{Smalltalk-80: The Language and Its Implementation}
    \end{columns}
\end{frame}

\begin{frame}{Java}
    \begin{columns}
        \column{0.3\textwidth}
        \begin{figure}
            \centering
            \includegraphics[width=0.8\textwidth]{../images/books/java-nutshell}
        \end{figure}
        \column{0.7\textwidth}
        A class is a collection of data fields that hold values and
        methods that operate on those values.\\
        -- \textbf{Ben Evans}, \emph{Java in a Nutshell}
    \end{columns}
\end{frame}

\begin{frame}{Eckel}
    \begin{columns}
        \column{0.3\textwidth}
        \begin{figure}
            \centering
            \includegraphics[width=0.8\textwidth]{../images/books/eckel}
        \end{figure}
        \column{0.7\textwidth}
        Each object looks quite a bit like a little computer --- it
        has a state, and it has operations that you can ask it to
        perform.\\
        -- \textbf{Bruce Eckel}, \emph{Thinking in Java}
    \end{columns}
\end{frame}

\begin{frame}{West}
    \begin{columns}
        \column{0.3\textwidth}
        \begin{figure}
            \centering
            \includegraphics[width=0.8\textwidth]{../images/books/object-thinking}
        \end{figure}
        \column{0.7\textwidth}
        An object is the equivalent of the quanta from which
        the universe is constructed.\\
        -- \textbf{David West}, \emph{Object Thinking}
    \end{columns}
\end{frame}

\section{Ba thành phần khó của OOP}

\begin{frame}{Phương thức tĩnh}
    \begin{columns}
        \column{\textwidth}
        \begin{figure}
            \centering
            \frame{\includegraphics[width=0.7\textwidth]{static1}}
        \end{figure}
        % \column{0.5\textwidth}
        \vspace{-0.5cm}
        \begin{figure}
            \centering
            \frame{\includegraphics[width=0.6\textwidth]{static2}}
        \end{figure}
    \end{columns}
\end{frame}

\begin{frame}{Tính bất biến (immutability)}
    \begin{columns}
        \column{\textwidth}
        \begin{figure}
            \centering
            \frame{\includegraphics[width=0.6\textwidth]{mutable}}
        \end{figure}
        % \column{0.5\textwidth}
        \vspace{-0.5cm}
        \begin{figure}
            \centering
            \frame{\includegraphics[width=0.55\textwidth]{immutable}}
        \end{figure}
    \end{columns}
\end{frame}

\begin{frame}{Lợi ích của tính bất biến}
    \begin{figure}
        \centering
        \includegraphics[width=0.9\textwidth]{immutability-benefits}
    \end{figure}
    \centering\qrcode[height=0.7in]{https://www.yegor256.com/2014/06/09/objects-should-be-immutable.html}
\end{frame}

\begin{frame}{Tham chiếu NULL}
    \begin{figure}
        \centering
        \includegraphics[width=0.9\textwidth]{null}
    \end{figure}
    \centering\qrcode[height=0.7in]{https://www.yegor256.com/2014/05/13/why-null-is-bad.html}
\end{frame}

\begin{frame}{Đối tượng NULL}
    \begin{figure}
        \centering
        \includegraphics[width=0.9\textwidth]{null-object}
    \end{figure}
\end{frame}

\begin{frame}{Fail Fast vs. Fail Safe}
    \begin{figure}
        \centering
        \includegraphics[width=0.9\textwidth]{fail-fast}
    \end{figure}
    \centering\qrcode[height=0.7in]{https://www.yegor256.com/2015/08/25/fail-fast.html}
\end{frame}

\section{Biểu đồ lớp}

\begin{frame}{Biểu đồ lớp}
    \begin{figure}
        \centering
        \includegraphics[width=0.7\textwidth]{ddd}
    \end{figure}
\end{frame}

\begin{frame}{Tài liệu tham khảo}
    \begin{columns}
        \column{0.5\textwidth}
        \begin{figure}
            \centering
            \includegraphics[width=0.4\textwidth]{../images/books/object-thinking}
            \caption{David West, \emph{Object Thinking}}
        \end{figure}
        \column{0.5\textwidth}
        \begin{figure}
            \centering
            \includegraphics[width=0.4\textwidth]{../images/books/ddd}
            \caption{Eric Evans,
            \emph{Domain-Driven Design: Tackling Complexity in the Heart of Software}}
        \end{figure}
    \end{columns}
\end{frame}

\begin{frame}{Tài liệu tham khảo}
    \begin{columns}
        \column{0.5\textwidth}
        \begin{figure}
            \centering
            \includegraphics[width=0.4\textwidth]{../images/books/elegant-objects-1}
            \caption{Yegor Bugayenko, \emph{Elegant Objects, vol. 1}}
        \end{figure}
        \column{0.5\textwidth}
        \begin{figure}
            \centering
            \includegraphics[width=0.4\textwidth]{../images/books/elegant-objects-2}
            \caption{Yegor Bugayenko, \emph{Elegant Objects, vol. 2}}
        \end{figure}
    \end{columns}
\end{frame}

\end{document}
