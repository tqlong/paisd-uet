\documentclass{beamer}

% Vietnamese support
\usepackage{fontenc}
\usepackage{qrcode}

\newcommand{\ul}[1]{\emph{#1}}

% Beamer theme
\usetheme{Berlin}

\title{Thực hành phát triển hệ thống Trí tuệ nhân tạo\\
Phân tích thiết kế hướng đối tượng (OOAD)
}
\author{Trần Quốc Long}
\institute{Trường ĐH Công nghệ, ĐHQG Hà Nội}
\date{\today}

\begin{document}

\begin{frame}
    \titlepage
\end{frame}

\begin{frame}{Mục lục}
    \tableofcontents
\end{frame}

\section{Lập trình hướng đối tượng}

\begin{frame}{Các nguyên tắc trong lập trình hướng đối tượng}
    \begin{columns}
    \column{0.5\textwidth}
    \begin{itemize}
        \item \ul{Trừu tượng} (Abstraction)
        \item \ul{Đóng gói} (Encapsulation)
        \item \ul{Mô-đun} (Modularity)
        \item \ul{Phân cấp} (Hierarchy)
    \end{itemize}
    \column{0.5\textwidth}
    \begin{itemize}
        \item \ul{Đa hình} (Polymorphism)
        \item \ul{Thừa kế} (Inheritance)
        \item \ul{Hợp thành} (Composition)
        \item \ul{Giao tiếp} (Delegation)
        \item \ul{Phân loại} (Subtyping)
        \item \ul{Ẩn dữ liệu} (Data Hiding)
        \item \ul{Tách biệt quan tâm} (Separation of Concerns)
    \end{itemize}
    \end{columns}
\end{frame}

\begin{frame}{Trừu tượng hoá}
    \begin{figure}
        \centering
        \includegraphics[width=\textwidth]{oop-vs-pp}
    \end{figure}
\end{frame}

\begin{frame}{Trừu tượng và đa hình}
    \begin{figure}
        \centering
        \includegraphics[width=0.6\textwidth]{java-abstraction}
    \end{figure}
\end{frame}

\begin{frame}{Trừu tượng và đa hình}
    \begin{figure}
        \centering
        \includegraphics[width=0.4\textwidth]{java-abstraction2}
    \end{figure}
\end{frame}

\begin{frame}{Đóng gói}
    \begin{figure}
        \centering
        \includegraphics[width=0.9\textwidth]{encapsulation}
    \end{figure}
\end{frame}

\begin{frame}{Đóng gói và Ẩn dữ liệu}
    \begin{figure}
        \centering
        \includegraphics[width=0.9\textwidth]{encapsulation2}
    \end{figure}
\end{frame}

\begin{frame}{Đóng gói và Ẩn dữ liệu}
    \begin{figure}
        \centering
        \includegraphics[width=0.6\textwidth]{naked1}
    \end{figure}
    \begin{figure}
        \centering
        \includegraphics[width=0.6\textwidth]{naked2}
    \end{figure}
\end{frame}

\begin{frame}{Mô-đun}
    \begin{figure}
        \centering
        \includegraphics[width=\textwidth]{modularity}
    \end{figure}
\end{frame}

\begin{frame}{Phân cấp}
    \begin{figure}
        \centering
        \includegraphics[width=0.9\textwidth]{hierarchy}
    \end{figure}
\end{frame}

\begin{frame}{Thừa kế và Hợp thành}
    \begin{figure}
        \centering
        \includegraphics[width=0.9\textwidth]{inheritance-versus-composition}
    \end{figure}
\end{frame}

\begin{frame}{Thừa kế và Hợp thành}
%% create table comparion of inheritance and composition
%% when, why, how
\begin{table}
    \centering
    \begin{tabular}{|p{2cm}|p{3.5cm}|p{3.5cm}|}
        \hline
        & \textbf{Thừa kế} & \textbf{Hợp thành} \\
        \hline
        \textbf{Khi nào} & Quan hệ ``is-a'' \newline ràng buộc chặt 
        & Quan hệ ``has-a'' \newline ràng buộc lỏng\\
        \hline
        \textbf{Khi nào không} &
            Chỉ dùng một phần chức năng của lớp cha\newline
            Lớp cha thay đổi thường xuyên\newline
            Cần thừa kế từ nhiều lớp cha
        &  Thừa kế giảm nhiều trùng lặp hơn \newline
        Chức năng cần được thừa kế và nạp chồng \\
        \hline
        \textbf{Tại sao} & Tái sử dụng mã nguồn & Tái sử dụng chức năng \\
        \hline
    \end{tabular}
\end{table}

\end{frame}

\section{Nguyên tắc SOLID}

\begin{frame}{Các nguyên tắc SOLID}
    \begin{columns}
        \column{0.3\textwidth}
        \begin{figure}
            \centering
            \includegraphics[width=0.8\textwidth]{../images/books/agile-martin}
        \end{figure}
        \column{0.7\textwidth}
            \only<1>{
                \begin{block}{}
                    Trong Agile, thái độ của nhà phát triển đối với
                    thiết kế phần mềm giống như thái độ của các bác
                    sĩ phẫu thuật đối với quy trình vô trùng.
                    Quy trình vô trùng điều kiện tiên quyết của phẫu
                    thuật. Nếu không có nó, nguy cơ nhiễm trùng sẽ
                    quá cao để chịu đựng. Các nhà phát triển Agile
                    cũng cảm thấy tương tự đối với thiết kế của họ.\\
                    \textbf{Robert Martin}, \emph{Agile Software Development. Principles, Patterns, and Practices}
                \end{block}
            }
    \end{columns}
\end{frame}

\begin{frame}{Các nguyên tắc SOLID}
    \begin{itemize}
        \item \textbf{\color{red}S}ingle Responsibility Principle - đơn trách nhiệm
        \item \textbf{\color{red}O}pen Close Principle - mở đóng
        \item \textbf{\color{red}L}iskov Substitution Principle - thay thế Liskov
        \item \textbf{\color{red}I}nterface Segregation Principle - phân chia giao diện
        \item \textbf{\color{red}D}ependency Inversion Principle - đảo ngược phụ thuộc
    \end{itemize}
\end{frame}

\begin{frame}{Nguyên tắc đơn trách nhiệm}
    \begin{itemize}
        \item Một lớp/mô-đun chỉ nên có một lý do để thay đổi
        \item Mỗi lớp/mô-đun chỉ nên thực hiện một chức năng cụ thể
        và phải đóng gói chức năng đó
        \item Mỗi lớp/mô-đun chỉ nên thực hiện một vai trò
    \end{itemize}

    \begin{table}
        \centering
        \begin{tabular}{|p{4.5cm}|p{4.5cm}|}
            \hline\vspace{-0.5cm}
            \begin{center}\textbf{Nên dùng}\end{center} 
            & \vspace{-0.5cm}\begin{center}\textbf{Không nên dùng}\end{center} \\
            \hline
            \vspace{-0.3cm}
            \begin{itemize}
                \item Lớp có nhiều chức năng không liên quan
                \item Lớp quá lớn
                \item Lớp có bộ phận hay thay đổi
            \end{itemize}
            & \vspace{-0.3cm}\begin{itemize}
                \item Phân tách lớp khiến code phức tạp hơn
                \item Phạm vi của lớp nhỏ, các thành phần liên quan mạnh
                \item Hiệu suất giảm quá nhiều
            \end{itemize} \\
            \hline
        \end{tabular}
    \end{table}
\end{frame}

\begin{frame}{Nguyên tắc đơn trách nhiệm}
    \begin{figure}
        \centering
        \includegraphics[width=\textwidth]{srp}
    \end{figure}
\end{frame}

\begin{frame}{Nguyên tắc mở đóng}
    \Large Một lớp phải \textbf{\color{red}mở} để mở rộng
    nhưng \textbf{\color{red}đóng} đối với sửa đổi
    \vspace{-0.5cm}
    \begin{figure}
        \centering
        \includegraphics[width=0.7\textwidth]{ocp}
    \end{figure}
\end{frame}

\begin{frame}{Nguyên tắc thay thế Liskov}
    Nếu với mỗi đối tượng $o_1$ thuộc lớp $S$ 
    có đối tượng $o_2$ thuộc lớp $T$ sao cho mọi chương trình
    $P$ viết dựa trên $T$ mà hành vi của $P$ không thay đổi
    khi $o_1$ thay thế $o_2$ thì $S$ là lớp con của $T$
    -- \textbf{Barbara Liskov}
    % \vspace{-0.3cm}
    \begin{figure}
        \centering
        \includegraphics[width=0.7\textwidth]{lsp}
    \end{figure}
\end{frame}

\begin{frame}{Duck Typing}
    \Large Nếu nó trông như vịt, bơi như vịt và 
    kêu như vịt, thì chắc nó là vịt
    \begin{figure}
        \centering
        \includegraphics[width=0.7\textwidth]{duck}
    \end{figure}
\end{frame}

\begin{frame}{Nguyên tắc phân chia giao diện}
    \Large Đối tượng khách (client) không nên bị ép buộc
    phụ thuộc vào các phương thức mà nó không sử dụng
    -- \textbf{Robert Martin}
    \begin{figure}
        \centering
        \includegraphics[width=0.5\textwidth]{isp}
    \end{figure}
\end{frame}

\begin{frame}{Nguyên tắc đảo ngược phụ thuộc}
    (a) Các module cấp cao không nên phụ thuộc vào
    các module cấp thấp. Cả hai nên phụ thuộc vào các giao diện
    (b) Các giao diện không nên phụ thuộc vào chi tiết.
    Chi tiết nên phụ thuộc vào giao diện
    \begin{figure}
        \centering
        \includegraphics[width=\textwidth]{dip}
    \end{figure}
\end{frame}

\section{Một số nguyên tắc khác}

\begin{frame}{Thiết kế dựa trên hợp đồng}
    \begin{figure}
        \centering
        \includegraphics[width=0.6\textwidth]{contracts}
    \end{figure}
\end{frame}

\begin{frame}{Tránh trùng lặp}
    \Large Don't Repeat Yourself (DRY)
    \begin{figure}
        \centering
        \includegraphics[width=0.5\textwidth]{dry}
    \end{figure}
\end{frame}

\begin{frame}{Giản lược tính năng}
    \Large You Ain't Gonna Need It (YAGNI)
    \begin{figure}
        \centering
        \includegraphics[width=0.7\textwidth]{yagni}
    \end{figure}
\end{frame}

\begin{frame}{Đảo ngược kiểm soát}
    \Large Inversion of Control (IoC)
    \begin{figure}
        \centering
        \includegraphics[width=0.9\textwidth]{ioc}
    \end{figure}
\end{frame}

\begin{frame}{OOP và Lập trình hàm}
    \begin{figure}
        \centering
        \includegraphics[width=0.9\textwidth]{oop-vs-fp}
    \end{figure}
\end{frame}

\begin{frame}{Nhược điểm của OOP}
    \begin{columns}
        \column{0.5\textwidth}
        \begin{figure}
            \centering
            \includegraphics[width=0.9\textwidth]{issues}
        \end{figure}
        \column{0.5\textwidth}
        \Large What's Wrong With Object-Oriented Programming? (2016)
        \qrcode[height=1in]{https://www.yegor256.com/2016/08/15/what-is-wrong-object-oriented-programming.html}
    \end{columns}
\end{frame}

\begin{frame}{Tài liệu tham khảo}
    \begin{columns}
        \column{0.5\textwidth}
        \begin{figure}
            \centering
            \includegraphics[width=0.4\textwidth]{../images/books/clean-code}
            \caption{Robert C. Martin, \emph{Clean Code: A Handbook of Agile Software Craftsmanship}}
        \end{figure}
        \column{0.5\textwidth}
        \begin{figure}
            \centering
            \includegraphics[width=0.4\textwidth]{../images/books/booch-book}
            \caption{Grady Booch và cộng sự,
            \emph{Object-Oriented Analysis and Design with Applications}}
        \end{figure}
    \end{columns}
\end{frame}

\begin{frame}{Tài liệu tham khảo}
    \begin{columns}
        \column{0.5\textwidth}
        \begin{figure}
            \centering
            \includegraphics[width=0.4\textwidth]{../images/books/stroustrup}
            \caption{Bjarne Stroustrup, \emph{Programming: Principles and Practice Using C++}}
        \end{figure}
        \column{0.5\textwidth}
        \begin{figure}
            \centering
            \includegraphics[width=0.4\textwidth]{../images/books/head-first}
            \caption{Brett McLaughlin và cộng sự, \emph{Head First Object-Oriented Analysis and Design: A Brain Friendly Guide to OOA\&D}}
        \end{figure}
    \end{columns}
\end{frame}

\end{document}
